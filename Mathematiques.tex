\documentclass[13pt, twoside, a4paper, french]{report}

\usepackage{ficheslib}
\usepackage{stmaryrd}
\newcommand*{\getSubject}{Mathématiques}


\begin{document}
    \title{\getSubject}
    \author{Clément \sn{Grennerat}}
    \date{Septembre 2022}
    \tableofcontents

    \chapter{Logique et raisonnement}\label{ch:logique-et-raisonnement}
        \section{Logique}\label{sec:logique}
  
  \subsection{Connecteurs logiques $\neg$, $\land$, $\lor$, ($\oplus$ ou $\veebar$)}\label{subsec:connecteurs-logiques}
    
    \fourCol[33][18][19]{
      
      \textbf{La négation $\neg$}, pour les ensembles, correspond au complémentaire du sous ensemble $A$ dans $E$, noté $\complement_{E}A$, $\overline{A}$ ou $A^C$ \\
      Négation $\neq$ Contraire (Non jeune $\neq$ vieux)
    }{
      \textbf{La conjonction $\land$} pour ET, correspond à l'intersection $\cap$.
    }{
      \textbf{La disjonction $\lor$} OU, correspond à l'union $\cup$.
    }{
      \textbf{La disjonction exclusive $\oplus$ ou $\veebar$} XOR, l'une des deux vrais et l'autre nécessairement fausse.\\
      Correspond à la différence symétrique $\Delta$.
    }
    
    \vspace*{7pt}
    
    \twoCol{
      \makebox[3cm]{\textbf{Règles de Morgan :}} $\neg(P \lor Q) \Leftrightarrow (\neg P) \land (\neg Q)$ \\
      \makebox[3cm]{} $\neg(P \land Q) \Leftrightarrow (\neg P) \lor (\neg Q)$
    }{
      \makebox[3cm]{\textbf{Distributivité :}} $(P \lor Q) \land R \Leftrightarrow (P \land R) \lor (Q \land R)$ \\
      \makebox[3cm]{} $(P \land Q) \lor R \Leftrightarrow (P \lor R) \land (Q \lor R)$
    }
    
    \makebox[3cm]{\textbf{Produit cartésien :}} $A \times B = \{(a, b),\ a \in A\ et\ b \in B\}$
  
  \subsection{Connecteurs logiques $\Rightarrow$ et $\Leftrightarrow$}\label{subsec:connecteurs-logiques-2}
    
    \fourCol[35][19][27]{
      \textbf{L'implication} de P à Q n'est fausse que lorsque P est vraie et Q fausse.\\
      \makebox[2.3cm][r]{\textbf{Correspond à :}} $\neg P \lor Q$\\
      \makebox[2.3cm][r]{\textbf{Négation :}} $P \land \neg Q$\\
      \makebox[2.3cm][r]{\textbf{Réciproque :}} $Q \Rightarrow P$\\
      \makebox[2.3cm][r]{\textbf{Contraposée :}} $(\neg Q) \Rightarrow (\neg P)$\\\;(même valeur de vérité).
    }{
      \begin{tabular}[t]{|c|c|c|}
        \hline
        \headRow
        P                     & Q                     & $P \Rightarrow Q$     \\ \hline
        \cellcolor{green!15}V & \cellcolor{green!15}V & \cellcolor{green!15}V \\ \hline
        \cellcolor{green!15}V & \cellcolor{red!15}F   & \cellcolor{red!15}F   \\ \hline
        \cellcolor{red!15}F   & \cellcolor{green!15}V & \cellcolor{green!15}V \\ \hline
        \cellcolor{red!15}F   & \cellcolor{red!15}F   & \cellcolor{green!15}V \\ \hline
      \end{tabular}%
    }{
      \textbf{L'équivalence} de P et Q n'est vraie que si P et Q ont même valeur de vérité.\\
      L'équivalence a la même valeur de vérité que la double implication (nécessaire et suffisante).
    }{
      \begin{tabular}[t]{|c|c|c|}
        \hline
        \headRow
        P                     & Q                     & $P \Leftrightarrow Q$ \\ \hline
        \cellcolor{green!15}V & \cellcolor{green!15}V & \cellcolor{green!15}V \\ \hline
        \cellcolor{green!15}V & \cellcolor{red!15}F   & \cellcolor{red!15}F   \\ \hline
        \cellcolor{red!15}F   & \cellcolor{green!15}V & \cellcolor{red!15}F   \\ \hline
        \cellcolor{red!15}F   & \cellcolor{red!15}F   & \cellcolor{green!15}V \\ \hline
      \end{tabular}
    }
  
  \subsection{Quantificateurs}\label{subsec:quantificateurs}
    
    $\forall x \in A, P(x)$ : quel que soit $x$ élément de $A$ (ou pour tout $x$ appartenant à $A$). \hfill \textbf{Négation :} $\exists x \in A, \neg P(x)$\\
    $\exists x \in A, P(x)$ : il existe au moins un élément $x$ de $A$ vérifiant $P$. \hfill \textbf{Négation :} $\forall x \in A, \neg P(x)$\\
    $\exists !x \in A, P(x)$ : il y a existence et \textbf{unicité} de l'élément $x$ dans $A$ vérifiant $P$.


\section{Méthodes de raisonnement}\label{sec:methodes-de-raisonnement}
  
  \begin{multicols}{2}
    \raggedcolumns
    
    \subsection{Raisonnement direct}\label{subsec:raisonnement-direct}
      
      (Par déduction ou hypothèse auxiliaire).\\
      Hypothèse $\Rightarrow$ Conclusion.
    
    \subsection{Raisonnement par disjonction des cas}\label{subsec:raisonnement-par-disjonction-des-cas}
      
      Raisonnement direct dans lequel l'hypothèse peut se décomposer en plusieurs sous hypothèses :\\
      $H \Leftrightarrow H_1 \land H_2$\\
      On montre ensuite $H_1 \Rightarrow C$ et $H_2 \Rightarrow C$
    
    \subsection{Raisonnement par contraposition}\label{subsec:raisonnement-par-contraposition}
      
      $(H \Rightarrow C) \Leftrightarrow (\neg C \Rightarrow \neg H)$\\
      \textit{Exemple :} La contraposée de \og $\forall n \in \mathbb{Z}$, si $n$ est pair, alors $n^2$ est pair. \fg\ est \og Si $n^2$ est impair, alors $n$ est impair \fg
    
    \subsection{Raisonnement par l'absurde}\label{subsec:raisonnement-par-l'absurde}
      
      On suppose le contraire et on montre que cela vient contredire une proposition vraie.\\
      \textit{Exemple :} Démontrons que $\sqrt{2}$ est irrationnel $\rightarrow$ supposons que $\sqrt{2}$ est rationnel.
      On aurait donc $2 = \frac{p^2}{q^2}$ avec $(p, q) \in \mathbb{N}^2$ et $p$ et $q$ premiers entre eux.
      Donc $p^2 = 2q^2$ alors $p$ pair, $p = 2p'$ d'où $q^2 = 2p'^2$ donc $q$ serait pair.
      Ainsi, p et q ne seraient pas premiers entre eux, on a bien une contradiction.
    
    \subsection{Raisonnement par contre-exemple}\label{subsec:raisonnement-par-contre-exemple}
      
      Il suffit de trouver un contre exemple pour prouver qu'une propriété est fausse.
    
    \subsection{Raisonnement par récurrence simple}\label{subsec:raisonnement-par-recurrence}
      
      - On montre que $P(0)$ est vraie (propriété initialisée ou fondée)\\
      - On suppose $P(n)$ et on montre $P(n+1)$ (hérédité)
    
    \subsection{Raisonnement par analyse-synthèse}\label{subsec:raisonnement-par-analyse-synthese}
      
      - Analyse : on établit une liste de potentielles solutions parmi lesquelles toutes les solutions réelles sont nécessairement incluses.\\
      - Synthèse : pour chacune de ces solutions,  on détermine si elles sont viables ou non.
  
  \end{multicols}
  
  



    \chapter{Généralités sur les fonctions}\label{ch:generalites-sur-les-fonctions}
        \section{Généralités}\label{sec:generalites}
  
  \subsection{Différence entre applications et fonctions}\label{subsec:difference-entre-applications-et-fonctions}

    Une application est une relation entre deux ensembles.\\
    Une fonction est une application d'une partie $D_f$ d'un ensemble de départ. $D_f$ est appelé ensemble de définition.\\
    Une fonction n'est donc pas forcément définie sur l'entièreté de l'ensemble de départ.
    
    \vspace*{5pt}
    
    \fourCol[25][26][20]{
      \subsection{Image directe}\label{subsec:image-directe}
      \textbf{Ensemble des images} de $A$ par $f$ :\\ $f(A) = \{f(x)\ /\ x \in A\}$
    }{
      \subsection{Image réciproque}\label{subsec:image-reciproque}
      
      \textbf{Ensemble des antécédents} d'un ensemble.\\
      C'est l'image directe de l'ensemble par $f^{-1}$.
    }{
      \subsection{Restriction}\label{subsec:restriction-d'une-application}
      
      On note $f_{|A}(x) = f(x)$ pour tout $x \in A$, avec $A \subset D_f$.
    }{
      
      \subsection{Composition}\label{subsec:composition}
      $f \circ g(x) = f(g(x))$\\
      \textbf{Associative} mais non commutative :\\
      $(f \circ g) \circ h = f \circ (g \circ h)$
    }
    
    \vspace*{5pt}
    
    \twoCol[55]{
      \subsection{Injectivité}\label{subsec:injectivite}
      
      \textbf{Au plus} un antécédent.\\
      Fonction injective : $f(x_1) = f(x_2) \Rightarrow x_1 = x_2$.\\
      Non injective : $\exists (x_1, x_2) \in D_f^2\ /\ f(x_1) = f(x_2) \land x_1 \neq x_2$
    }{
      \subsection{Surjectivité}\label{subsec:surjectivite}
      
      \textbf{Au moins} un antécédent.\\
      Fonction surjective : $\forall y \in F,\ \exists x \in D_f\ /\ y = f(x)$.\\
      Non surjective : $\exists y \in F\ /\ \forall x \in D_f,\ y \neq f(x)$
    }
  
  \subsection{Bijectivité ou Réciprocité}\label{subsec:bijectivite-ou-reciprocite}
    À la fois injective et surjective : $\forall y \in F,\ \exists! x \in D_f\ /\ y = f(x)$.\\
    On note alors $x = f^{-1}(y)$ la bijection réciproque de f.


\section{Fonctions de $\mathbb{R}$ dans $\mathbb{R}$}\label{sec:fonctions-de-r-dans-r}
  
  \twoCol{
    
    \subsection{Sens de variation}\label{subsec:sens-de-variation}
    
    $f \circ g$ est :
    \begin{itemize}
      \item Croissante si $f$ et $g$ sont de même monotonie.
      \item Décroissante si $f$ et $g$ sont de sens de monotonies contraires.
    \end{itemize}
    
    \vspace*{7pt}
    Si $f$ est strictement monotone, alors elle est injective (au plus un antécédent par image).
  }{
    
    \subsection{Majorant et minorant}\label{subsec:majorant-et-minorant}
    
    \begin{outline}
      \1 \textbf{$\alpha$ maximum de A} si $\alpha$ est à la fois majorant et élément de A : $\alpha = \max(A)$ \;\; (resp. $\min(A)$).
      \1 \textbf{$\alpha$ borne supérieure de A} si $\alpha$ est le plus petit des majorants : $\alpha = \sup(A)$ \;\; (resp. $\inf(A)$).
      \2 si A est non vide non majoré : $\sup(A) = +\infty$
      \2 si A est non vide non minoré : $\inf(A) = -\infty$
    \end{outline}
    Si le maximum existe, il est égal à la borne supérieure.
  }
  
  \subsection{Parité}\label{subsec:parite}
    
    \begin{itemize}
      \item Si $f$ et $g$ sont paires, $f + g$ est paire (resp. impaires).
      \item Si $f$ et $g$ ont même parité, $f \times g$ est paire (resp. $\sfrac{f}{g}$).
      \item Si $f$ et $g$ ont des parités contraires, $f \times g$ est impaire (resp. $\sfrac{f}{g}$).
      \item Si $f$ est paire, $g \circ f$ est paire.
      \item Si $f$ est impaire, $g \circ f$ a la même parité que $g$.
    \end{itemize}
    
    \vspace*{7pt}
    
    \twoCol {
      
      \subsection{Périodicité}\label{subsec:periodicite}
      
      $T$-périodique si $\forall (x,x+T) \in D_f^2$,\; $f(x + T) = f(x)$.
      \begin{itemize}
        \item Si $f$ et $g$ sont $T$-périodiques,\\$f + g$ et $f \times g$ sont $T$-périodiques.
        \item Si $f$ est $T$-périodique,\\$g \circ f$ est $T$-périodique.
      \end{itemize}
    }{
      
      \subsection{Bijectivité et symétrie}\label{subsec:bijectivite-et-symetrie}
      
      \begin{itemize}
        \item Si $f$ est une bijection, $C_f$ et $C_{f^{-1}}$ sont symétriques par rapport à $y = x$.
        \item $x \mapsto f(-x)$ : symétrie par l'axe des ordonnées.
        \item $x \mapsto -f(x)$ : symétrie par l'axe des abscisses.
        \item $x \mapsto f(x + a)$ : translation de vecteur $-a \vec{\imath}$.
        \item $x \mapsto f(x) + a$ : translation de vecteur $b \vec{\jmath}$.
        \item $x \mapsto f(ax)$ : réduction/agrandissement sur axe x.
        \item $x \mapsto af(x)$ : réduction/agrandissement sur axe y.
      \end{itemize}
    }





    \chapter{Fonctions usuelles}\label{ch:fonctions-usuelles}
          \section{Fonction partie entière}\label{sec:fonction-partie-entiere}

    $E(x) = \lfloor x \rfloor$ (Plus grand entier inférieur ou égal à x).\\

    \begin{itemize}
      \item $E(x) \le x < E(x) + 1$ et $x-1 < E(x) \le x$.
      \item $E(x) = x \Leftrightarrow x \in \mathbb{Z}$.
      \item $\forall n \in \mathbb{Z},\ E(x + n) = E(x) + n$
    \end{itemize}


  \section{Fonction log, exp et puissances}\label{sec:fonction-log-exp-puissances}

    \subsection{Logarithme naturel / népérien}\label{subsec:logarithme-naturel-/-neperien}

      $\forall x > 0,\ \ln(a)$ vaut l'aire sous la courbe de $\dfrac{1}{x}$ entre $1$ et $a$.\\
      On note $e$ tel que $\ln(e) = 1$ (base du logarithme népérien).\\
      Sa bijection réciproque est la fonction exponentielle, dont l'unique dérivée vérifiant la condition initiale $f(0) = 1$ est elle même.\\

      \twoCol {

      \subsection{Fonctions logarithmes}\label{subsec:fonctions-logarithmes}

      $\log_b(a) = \dfrac{\ln a}{\ln b}$.\\
      $\log_e$ : népérien, $\log_2$ : binaire, $\log_{10}$ : décimal ($\log$).
      De même, $\exp_a(x) = a^x$.
    }{

      \subsection{Fonctions puissances}\label{subsec:fonctions-puissances}
      $X^a$ est bijective de réciproque $X^\frac{1}{a}$ ($a \neq 0$)\\
      Sur $\mathbb{R}_+^*$ :\\
      $X^a$ est croissante si $a > 0$ et décroissante si $a < 0$.
    }\\


  \section{Fonctions trigonométriques}\label{sec:fonctions-trigonometriques}

    \twoCol {
      \underline{Angles opposés :}\\
      \makebox[4cm][l]{$\cos(-\theta) = \cos(\theta)$} $\sin(-\theta) = -\sin(\theta)$.\\
      \underline{Angles supplémentaires :}\\
      \makebox[4cm][l]{$\cos(\pi - \theta) = - \cos(\theta)$} $\sin(\pi - \theta) = \sin(\theta)$\\
      \makebox[4cm][l]{$\cos(\pi + \theta) = - \cos(\theta)$} $\sin(\pi + \theta) = - \sin(\theta)$\\
      \underline{Angles complémentaires :}\\
      \makebox[4cm][l]{$\cos\left(\frac{\pi}{2} - \theta\right) = \sin(\theta)$} $\sin\left(\frac{\pi}{2} - \theta\right) = \cos(\theta)$\\
      \makebox[4cm][l]{$\cos\left(\frac{\pi}{2} + \theta\right) = - \sin(\theta)$} $\sin\left(\frac{\pi}{2} + \theta\right) = \cos(\theta)$
    }{
      \underline{Somme des angles :}\\
      $\cos(\theta + \varphi) = \cos \theta \cos \varphi - \sin \theta \sin \varphi$\\
      $\cos(\theta - \varphi) = \cos \theta \cos \varphi + \sin \theta \sin \varphi$\\
      $\sin(\theta + \varphi) = \sin \theta \cos \varphi + \cos \theta \sin \varphi$\\
      $\sin(\theta - \varphi) = \sin \theta \cos \varphi - \cos \theta \sin \varphi$\\

      $\cos(2\theta) = \cos^2 \theta - \sin^2 \theta = 2 \cos^2 \theta - 1 = 1 - 2 \sin^2 \theta$\\
      $\sin(2\theta) = 2 \sin \theta \cos \theta$
    }


  \section{Fonctions hyperboliques}\label{sec:fonctions-hyperboliques}

    \threeCol{
      \underline{Cosinus hyperbolique :}\\
      $\cosh(x) = \dfrac{e^x + e^{-x}}{2}$\\
      \begin{itemize}
        \item Fonction paire
        \item Strict. décroissante sur $\mathbb{R}_-$
        \item Strict. croissante sur $\mathbb{R}_+$
      \end{itemize}
    }{
      \underline{Sinus hyperbolique :}\\
      $\sinh(x) = \dfrac{e^x - e^{-x}}{2}$\\
      \begin{itemize}
        \item Fonction impaire
        \item Strictement croissante
      \end{itemize}
    }{
      \underline{Tangeante hyperbolique :}\\
      $\tanh(x) = \dfrac{\sinh(x)}{\cosh(x)}$\\
      \begin{itemize}
        \item Fonction impaire
        \item Strictement croissante
        \item Définie sur $\mathbb{R}$.
      \end{itemize}
    }
    \vspace{5pt}\\
    \underline{Lien avec sinus et cosinus :}
    \begin{itemize}
      \item $\cosh(a + b) = \cosh a \cosh b + \sinh a \sinh b$
      \item $\sinh(a + b) = \sinh a \cosh b + \cosh a \sinh b$
      \item $\cosh^2 a - \sinh^2 a = 1$
    \end{itemize}
    \vspace{5pt}\\
    $\cosh a = \cos(ia)$\\
    $\sinh a = -i\sin(ia)$


    \chapter{Polynômes}\label{ch:polynomes}
        \input{mathematiques/4-polynômes}


    \chapter{Espaces vectoriels}\label{ch:espaces-vectoriels}
        \section{Structure d'espace vectoriel}\label{sec:structure-d'espace-vectoriel}
  
  Un ensemble $E$ est un $\mathbb K$-espace vectoriel ($\mathbb K$ désigne $\mathbb{R}$ ou $\mathbb{C}$) si il est muni d'une loi interne $\textbf{+}$ et d'une loi externe \textbf{$\cdot$}.\\
  
  On redéfinit les lois de bases (commutativité et associativité de l'addition, élément neutre\ldots).


\section{Sous espaces vectoriels}\label{sec:sous-espaces-vectoriels}
  
  $F$ est un sous-espace vectoriel de $E$ si $F$ est une partie de $E$, avec $(E, +, \cdot)$ et $(F, +, \cdot)$ des $\mathbb K$-espace vectoriel.\\
  
  \textbf{Stabilité par combinaison linéaire :}
  \begin{equation}
    F \text{ est un s.e.v de } E \iff
    \begin{cases}
      F \neq \emptyset \text{ (Un s.e.v. contient toujours le vecteur nul)}\\
      \forall (\alpha, \beta, \vec u, \vec v) \in \mathbb{K}^2 \times F^2,\ \alpha \vec u + \beta \vec v \in F\\
      \text{(F est stable par combinaison linéaire)}
    \end{cases}\label{eq:equation2}
  \end{equation}
  
  L'intersection de deux s.e.v est un s.e.v.\ Ce n'est pas le cas pour l'union.\\
  
  \textbf{Sous-espace vectoriel engendré par A} ($Vect(A)$) : c'est le plus petit s.e.v. de $E$ contenant $A$.\\
  Par convention : $Vect(\emptyset) = {\vec{0_E}}$\\
  On dit que $Vect(A)$ est constitué de toutes les combinaisons linéaires des vecteurs de $A$.


\section{Dimension d'un espace vectoriel}\label{sec:dimension-d'un-espace-vectoriel}
  
  
  
  


    \chapter{Applications linéaires}\label{ch:applications-lineaires}
        \section{L'espace vectoriel $\mathcal{L}(E, F)$}\label{sec:l'espace-vectoriel-$mathcal{l}(e-f)$}
  
  $f$ est une application linéaire si $f \in \mathcal{L}(E, F)$\\
  $\iff \forall (\alpha, \beta) \in \mathbb{K}^2, \forall (\vec u, \vec v) \in E^2, f(\alpha \cdot_E \vec u +_E  \beta \cdot_E \vec v) = \alpha \cdot_F f(\vec u) +_F \beta \cdot_F f(\vec v)$

  Toute réciproque d'une bijection linéaire, combinaison linéaire d'une application linéaire ou composée de deux applications linéaires \textbf{est linéaire}.\\
  Une application linéaire bijective de $E$ dans $F$ est un isomorphisme de $E$ sur $F$.
  $E$ et $F$ sont isomorphes.

\section{Image par une application linéaire}\label{sec:image-par-une-application-lineaire}


\section{Applications linéaires en dimension finie}\label{sec:applications-lineaires-en-dimension-finie}



    \chapter{Limites et continuité\\Comparaison de fonctions}\label{ch:limites-est-continuite---comparaison-de-fonctions}
        \input{mathematiques/7-limites-et-continuité}


    \chapter{Continuité}\label{ch:continuite}
        \section{Continuité et grand théorèmes}\label{sec:continuite-et-grand-theoremes}

    \subsection{Fonctions continues sur un intervalle}\label{subsec:fonctions-continues-sur-un-intervalle}

        $f$ est continue sur $I$ $\iff$ $f$ est continue en tout point de $I$.\\
        On démontre souvent la continuité sur un intervalle par opérations de fonctions continues.\\

        \textbf{Théorème des valeurs intermédiaires :}\\
        Si $f$ est continue sur un intervalle $[a,\ b]$, alors pour tout réel $k$ compris entre $f(a)$ et $f(b)$, il existe un réel $c$ dans l'intervalle $[a,\ b]$ tel que $f(c) = k$.\\

        \underline{Corollaire 1 :} Si $f(a)$ et $f(b)$ sont de signe opposés, alors il existe un réel $c$ dans l'intervalle $[a,\ b]$ tel que $f(c) = 0$.\\
        \underline{Corollaire 2 :} L'image de $[a,\ b]$ est un intervalle $[f(a),\ f(b)]$.\\

        \textbf{Définition d'un segment :} C'est un intervalle fermé borné, du type $[a,\ b]$\\

        \textbf{Théorème des valeurs extrêmes :}\\
        Une fonction continue sur un segment est bornée et atteint ses bornes.\\
        \underline{Corollaire :} Si $f$ est continue, l'image d'un segment est un segment.\\

    \subsection{Continuité et bijection}\label{subsec:continuite-et-bijection}

        Si $f$ est continue et strictement monotone sur un intervalle $I$, alors :
        \begin{itemize}
            \item $f$ est une bijection de $I$ sur $f(I)$.
            \item $f^{-1}$ est continue et strictement monotone sur $f(I)$, de même sens de variation que $f$.
        \end{itemize}


\section{Application : Fonctions trigonométriques}\label{sec:application---fonctions-trigonometriques}

    \subsection{La fonction arc sinus}\label{subsec:la-fonction-arc-sinus}

        Restriction de sinus sur $[-\frac{\pi}{2},\ \frac{\pi}{2}]$ : strictement croissante, réalise une bijection sur $[-1,\ 1]$.\\
        Bijection réciproque : arcsin, continue et croissante sur $[-1,\ 1]$.

    \subsection{La fonction arc cosinus}\label{subsec:la-fonction-arc-cosunus}

        Restriction de cosinus sur $[0,\ \pi]$ : strictement décroissante, réalise une bijection sur $[-1,\ 1]$.\\
        Bijection réciproque : arccos, continue et strictement décroissante sur $[-1,\ 1]$.

    \subsection{La fonction arc tangente}\label{subsec:la-fonction-arc-tangente}

        Restriction de tangente sur $[-\frac{\pi}{2},\ \frac{\pi}{2}]$ : strictement croissante, réalise une bijection sur $\mathbb{R}$.\\
        Bijection réciproque : arctan, continue et croissante sur $\mathbb{R}$.

        \twoCol {

            \subsection{Relations fondamentales}\label{subsec:relations-fondamentales}

            $\cos(\arcsin(x)) = \sqrt{1-x^2}$\\
            $\sin(\arccos(x)) = \sqrt{1-x^2}$\\
            $\arcsin(x) + \arccos(x) = \dfrac{\pi}{2}$\\

            $\arctan(x) + \arctan\left(\frac{1}{x}\right) =$ \begin{cases}
                                                                 \dfrac{\pi}{2}&\re{ si }x > 0\\
                                                                 -\dfrac{\pi}{2}&\re{ si }x < 0
            \end{cases}\label{eq:equation3}
        }{

            \subsection{Equivalents usuels}\label{subsec:equivalents-usuels}

            $\arccos(x) \underset{0}{\sim} \dfrac{\pi}{2}$\\
            $\arcsin(x) \underset{0}{\sim} x$\\
            $\arctan(x) \underset{0}{\sim} x$\\
            $\arctan(x) \underset{+\infty}{\sim} \dfrac{\pi}{2}$\\
            $\arctan(x) \underset{-\infty}{\sim} -\dfrac{\pi}{2}$\\


        }







    \chapter{Dérivation}\label{ch:derivation}
        \section{Dérivées successives}\label{sec:deriv-es-successives}

    Fonction de classe $\mathcal{C}^n$ : $n$ fois dérivable et $f^{(n)}$ continue.\\
    Une somme, produit ou quotient de fonctions $\mathcal{C}^n$ reste $\mathcal{C}^n$ sur un intervalle $I$.\\
    $\exp$, $\ln$, $\sin$, $\cos$, polynômes et puissances sont $\mathcal{C}^{+\infty}$.


\section{Dérivabilité des fonctions réciproques}\label{sec:derivabilite-des-fonctions-reciproques}

    Soit $f$ continue strictement monotone. $f$ réalise donc une bijection.\\

    Si $f$ est dérivable en $a$ et $f'(a) \neq 0$, alors $f^{-1}$ est dérivable en $f(a) = b$.\\

    On a alors $(f^{-1})'(b) = \dfrac{1}{f'(a)} = \dfrac{1}{f' \circ f^{-1}(b)}$\\

    Cela permet de démontrer que $\arcsin'(x) = \dfrac{1}{\sqrt{1 - x^2}}$, $\arccos'(x) = -\dfrac{1}{\sqrt{1 - x^2}}$ et $\arctan'(x) = \dfrac{1}{1 + x^2}$.


\section{Fonctions à valeurs complexes}\label{sec:fonctions-a-valeurs-complexes}

    On peut décomposer la fonction en deux fonctions à valeurs réelles.
    \vspace{3pt}\\
    $f$ est dérivable si $\Re(f)$ et $\Im(f)$ le sont et $f'(x) = (\Re(f))' + i (\Im(f))'$.


\section{Théorème des accroissements finis (TAF)}\label{sec:theoreme-des-accroissements-finis-(taf)}

    \underline{Théorème de Rolle :}\\
    Soit $f$ continue sur $[a,\ b]$ et dérivable sur $]a,\ b[$ tel que $f(a) = f(b)$,\\
    alors il existe $c \in\ ]a,\ b[$ tel que $f'(c) = 0$.\\
    ($f$ admet au moins un extremum sur $]a,\ b[$)\\

    \underline{Théorème des accroissements finis :}\\
    Soit $f$ continue sur $[a,\ b]$ et dérivable sur $]a,\ b[$,\\
    alors il existe $c \in\ ]a,\ b[$ tel que $f(b) - f(a) = f'(c) (b - a)$.\\
    On note aussi $f'(c) = \dfrac{f(b) - f(a)}{b - a}$


\section{Applications du TAF}\label{sec:applications-du-taf}

    \subsection{Sens de variation d'une fonction et signe de la dérivée}\label{subsec:sens-de-variation-d'une-fonction-et-signe-de-la-derivee}
        Le signe de la dérivée donne la monotonie d'une fonction continue dérivable et vice versa.\\
        \underline{Stricte monotonie} de $f$ continue : $f'$ de signe constant ne s'annulant qu'en un nombre fini de points.

        \twoCol {
            \subsection{Extrema d'une fonction dérivable}\label{subsec:extrema-d'une-fonction-derivable}
            Une fonction possède un extremum local en $a$ s'il existe un voisinage $\mathcal{V}$ de $a$ où $\forall x \in \mathcal{V},\ f(x) \le f(a)$.\\
            (ou $\ge$ pour un minimum local).\\

            Dans ce cas, on a nécessairement $f'(a) = 0$.\\
            (Pour une fonction dérvable).\\

            Un réel $a$ pour lequel $f'(a) = 0$ est un point critique de $f$.\\
            Ça n'est un extremum que si $f'$ change de signe.
        }{
            \subsection{Théorème de la limite dérivée}\label{subsec:theoreme-de-la-limite-derivee}
            \begin{itemize}
                \item Si $\limm{x_0} f'(x) = l$, alors $f$ dérivable en $x_0$ et $f'(x_0) = l$.
                \vspace{5pt}
                \item Si $l = \pm \infty$, alors $f$ n'est pas dérivable en $x_0$\\
                (tangeante verticale).
                \vspace{5pt}
                \item Si $f'$ a une limite à gauche et à droite différente en $a$, alors $f$ n'est pas dérivable en $a$.
                \vspace{5pt}
                \item Si $f'$ n'admet pas de limite en $a$, on ne peut rien conclure.
            \end{itemize}

        }



    \chapter{Compléments d'algèbre linéaire}\label{ch:complements-d-algebre-lineaire}
        \section{Somme de sous-espace vectoriels}\label{sec:somme-de-sous-espace-vectoriels}

    \subsection{Somme et somme directe}\label{subsec:somme-et-somme-directe}

        Le sev somme de $F$ et $G$ est $H = \{u_F + u_G,\ u_F \in F,\ u_G \in G\}$\\

        On a : $F + G = \Vect{F \cup G}$ ($F + G$ est le plus petit SEV engendré par les parties $F$ et $G$).\\
        $F + G$ est directe si $F \cap G = \{\vec 0 \}$ On note $F \oplus G$.\\
        Tout élément de $F \oplus G$ s'écrit de façon unique comme somme d'un élément de $F$ et de $G$.

    \subsection{Supplémentaires}\label{subsec:supplementaires}
        $F$ et $G$ sont supplémentaires dans $E$ $\iff$ $E = F \oplus G$ (donc si $F \cap G = \{0_E\}$)\\

        Un sev peut avoir une infinité de supplémentaires.

    \subsection{En dimension finie}\label{subsec:en-dimension-finie}

        \underline{Formule de Grassman :}\\
        Soit $E$ un ev de dimension finie et $F$ et $G$ deux sev de $E$ :
        \[\dim(F + G) = \dim(F) + \dim(G) - \dim(F \cap G)\]

        $\mathcal{B}_F \cap \mathcal{B}_G$ est une base de $E$ $\iff$ $F \oplus G = E$.\\
        Dans un ev de dimension finie $n$, tout sev $F$ admet des supplémentaires de dimension $n - \dim(F)$.


\section{Projections vectorielles}\label{sec:projections-vectorielles}

    \[E = F \oplus G \iff \forall \vec x \in E,\ \exists ! \vec x_G \in G,\ \exists ! \vec x_F \in F,\ \vec x = \vec x_F + \vec x_G\]

    Ainsi, on appelle projection vectorielle sur $F$ parallèlement à $G$ l'application $p : E \to E$ définie par $p(\vec x) = \vec x_F$.\\

    Ainsi, si $E = F \oplus G$,
    \begin{itemize}
        \item $p$ est linéaire (endomorphisme de $E$)
        \item $G = \Ker(p)$ et $F = \Im(p)$
        \item $F = \left\{\vec x \in E,\ p(\vec x) = \vec x\right\} = \Ker(p - id_E)$ ($F$ est l'espace des invariants de $p$)
    \end{itemize}
    \vspace{10px}
    En dimension finie, si un endomorphisme $p$ de $E$ est une projection, alors $\Ker(p) \oplus \Im(p) = E$.\\

    Si $p$ est une projection vectorielle, alors $p \circ p = p$.\\
    Réciproquement, si $p$ est un endomorphisme de $E$ tel que $p \circ p = p$, alors $p$ est une projection vectorielle sur $\Im(p)$ parallèlement à $\Ker(p)$.\\



    \chapter{Matrices}\label{ch:matrices}
        \section{Matrices et opérations}
    
    \subsection{Définition}
        
        Une matrice, notée $A \in \mathcal{M}_{n,p}$ est un tableau de scalaires à $n$ lignes et $p$ colonnes.\\
        Ses coefficients sont notés $a_{ij}$, $i$ étant la ligne et $j$ la colonne.\\
        
        \twoCol {
            \underline{Matrices particulières :}
            \begin{itemize}
                \item Matrice nulle : $0_\mathcal{M}_{n,p}$
                \item Matrice ligne : $n = 1$
                \item Matrice colonne : $p = 1$
                \item Matrice carrée : $n = p$, on note $A \in \mathcal{M}_{n}$
            \end{itemize}
        }{
            \underline{Matrices particulières carrées :}
            \begin{itemize}
                \item Triangulaire supérieure : $\forall i > j, a_{i,j} = 0$
                \item Triangulaire inférieure : $\forall i < j, a_{i,j} = 0$
                \item Diagonale : $\forall i \new j, a_{i,j} = 0$
                \item Identité : diagonale, avec coefs. égaux à 1.
            \end{itemize}
        }
    
    \subsection{Opérations sur les matrices}
        
        \twoCol {
            
            \underline{Somme et multiplication par un scalaire}
            
            \begin{itemize}
                \item \textbf{Loi interne $+$ :}\\Les coefficient $i,j$ s'ajoutent entre eux \\(matrices de même taille).\vspace{5pt}
                \item \textbf{Loi externe $\cdot$ :}\\Chaque coefficient est multiplié par le scalaire.
            \end{itemize}
        }{
            \underline{Produit de matrices}\\
            Soit $A \in \mathcal{M}_{n,p}$ et $B \in \mathcal{M}_{p,q}$, alors $AB = C \in \mathcal{M}_{n,q}$ :
            
            \[ c_{ij} = \sum_{k=1}^{p} a_{ik} b_{kj} \]
            On multiplie les lignes de $A$ par les colonnes de $B$.
        }
    
    \subsection{Règles de calcul}
        
        \underline{Puissances de matrices :}
        \begin{itemize}
            \item $A^k$ est définis uniquement pour $A \in \mathcal{M}_{n}$ et vaut $A \times A \times \dots \times A$ ($k$ fois).\vspace{5pt}
        \end{itemize}
        
        \underline{Produit matriciel :}
        \begin{itemize}
            \item On distingue la distributivité à gauche et à droite car $\mathcal{M}_{n,p}(\mathbb{K})$ n'est pas commutatif.
            \item La multiplication avec un scalaire est commutative : $\lambda \times A = A \times \lambda$.
            \item L'associativité est vérifiée : $A(B C) = (A B) C = ABC$.\vspace{5pt}
        \end{itemize}
        
        \underline{Matrices identité :}\\
        $I_n$ est l'élément neutre pour le produit matriciel.\vspace{5pt}
        
        \underline{Matrices particulières :}
        \begin{itemize}
            \item Le produit de deux matrices triangulaires supérieures est triangulaire supérieur
            \item Le produit de deux matrices triangulaires inférieures est triangulaire inférieur
            \item Le produit de deux matrices diagonales est diagonal\vspace{5pt}.
        \end{itemize}
        
        \underline{Binôme de Newton :}\\
        Si A et B commutent ($AB = BA$) :
        \[ (A + B)^n = \sum_{k=0}^{n} \binom{n}{k} A^{n-k} B^k \]


\section{Lien avec les applications linéaires}
    
    \subsection{Matrice d'une application linéaire}
        
        Une application linéaire $f$ est entièrement déterminée par la donnée de $f(e_1)$, \ldots, $f(e_p)$.
        On peut donc représenter une application linéaire par une matrice.\\
        
        La matrice de $f$ relativement aux bases $\mathcal{B}$ et $\mathcal{B}'$ a les colonnes formées des coordonnées des images des vecteurs de la base de départ $\mathcal{B}$ exprimés dans la base d'arrivée $\mathcal{B}'$.\\
        On note cette matrice $[f]_{\mathcal{B},\mathcal{B}'}$.\\
        
        Si $f$ est un endomorphisme, on peut choisir la même base \mathcal{B} de départ et d'arrivée et on note $[f]_{\mathcal{B}}$.
    
    \subsection{Image d'un vecteur par une application linéaire}
        
        On multiplie la matrice d'une application linéaire par
        $X = \begin{pmatrix}
                 x_1\\ \vdots\\ x_n
        \end{pmatrix}$
        pour obtenir le vecteur image de $X$ par $f$.\\
        On note $Y = AX$.
        
        
        \twoCol {
            
            \subsection{L'espace vectoriel $\mathcal{M}_{n,p}(\mathbb{K})$}
            
            $\mathcal{M}_{n,p}(\mathbb{K})$ est un $\mathbb{K}$-ev. donc :
            \begin{itemize}
                \item $[f + g]_{\mathcal{B},\mathcal{B}'} = [f]_{\mathcal{B},\mathcal{B}'} + [g]_{\mathcal{B},\mathcal{B}'}$
                \item $[\lambda f]_{\mathcal{B},\mathcal{B}'} = \lambda [f]_{\mathcal{B},\mathcal{B}'}$
            \end{itemize}
            \vspace{5pt}\\
            
            On définis ainsi un isomorphisme :\vspace{-8pt}\\
            \[\begin{cases}
                  \mathcal{L}(E,\ F) &\to \mathcal{M}_{n,p}(\mathbb{K})\\
                  f &\mapsto [f]_{\mathcal{B},\mathcal{B}'}
            \end{cases}\]\vspace{-8pt}\\
            Où $\dim(\mathcal{M}_{n,p}(\mathbb{K})) = np = \dim(F) \times \dim(E)$
        }{
            \subsection{Composées d'applications linéaires et produit matriciel}
            
            Soient $f\ :\ E \rightarrow F$ et $g\ :\ F \rightarrow G$ deux applications linéaires :
            
            \[[g \circ f]_{\mathcal{B}_E,\mathcal{B}_G} = [g]_{\mathcal{B}_F,\mathcal{B}_G} \times [f]_{\mathcal{B}_E,\mathcal{B}_F}\]
            
            On met en première la matrice de l'application appliquée en dernière.
        }


\section{Inversion de matrices}
    
    $A \in \mathcal{M}_n$ est inversible $\iff \exists A^{-1} \in \mathcal{M}_n\ /\ A^{-1}A = AA^{-1} = I_n$.\\
    Cela équivaut à dire que $A$ est la matrice d'une application linéaire bijective.\\
    
    Ainsi, on a $(A^{-1})^{-1} = A$ et $(AB)^{-1} = B^{-1} A^{-1}$.\\
    
    \underline{Calcul de l'inverse :} \vspace{4pt}\\
    Soit $A = [f]_\mathcal{B}$, alors $A^{-1} = [f^{-1}]_\mathcal{B}$.
    (L'écriture sous forme d'une AL correspond à celle matricielle.) \vspace{2pt}\\
    On résous donc le système $Y = AX$ : on exprime $X$ en fonction de $Y$ quelconque et on obtient $A^{-1}$.
    
    \twoCol{
        \section{Rang d'une matrice}
        
        Le rang d'une matrice $A$ est le rang des vecteurs colonne de $A$, et le rang de l'AL associée.\\
        
        Pour une matrice carré :\\ $A \in \mathcal{M}_n$ est inversible $\iff \rg(A) = n$.\\
        Pour une matrice rectangulaire :\\ $A \in \mathcal{M}_{n,p}$, $\rg(A) \le \min(n,p)$.
    }{
        \section{Transposition}
    
    }
    \vspace{1pt}\\


\section{Matrices de changements de bases}
    
    Matrice de passage de $\mathcal{B}$ à $\mathcal{B}'$ : $\re{Pass}_{\mathcal{B} \rightarrow \mathcal{B}'} = [\mathcal{B}']_\mathcal{B} =  [id_E]_{\mathcal{B}',\mathcal{B}}$.\\
    Elle exprime en colonne les coordonnés des vecteurs de $\mathcal{B}'$ exprimés dans la base $\mathcal{B}$.\\
    
    En multipliant par $[\mathcal{B}']_\mathcal{B}$ on transforme les coordonnés exprimés dans $\mathcal{B}'$ en coordonnés exprimés dans $\mathcal{B}$.\\
    Comme pour la composition de fonctions, on lit de droite à gauche : $[\mathcal{B''}]_{\mathcal{B}} = [\mathcal{B'}]_{\mathcal{B}} \times [\mathcal{B}'']_{\mathcal{B}'}$\\
    Pour un vecteur $x$ : $[x]_{\mathcal{B}} = [\mathcal{B'}]_\mathcal{B} \times [x]_{\mathcal{B}'}$
    \vspace{5pt}\\
    On a $([\mathcal{B}']_\mathcal{B})^{-1} = [\mathcal{B}]_{\mathcal{B}'}$ : il faudra souvent inverser la matrice de passage.
    \vspace{5pt}\\
    Pour changer la base de départ d'une AL, on multiplie à droite par la matrice de passage, et pour changer la base d'arrivée, on multiplie à gauche : $[f]_{\mathcal{B}',\mathcal{C}'} = [\mathcal{C}]_{\mathcal{C'}} \times [f]_{\mathcal{B},\mathcal{C}} \times [\mathcal{B'}]_{\mathcal{B}}$\\
    
    $A$ et $B$ sont équivalentes$\iff$sont la même AL dans des bases différentes$\iff A = Q^{-1} B P \iff \boldsymbol{\rg(A) = \rg(B)}$.\\
    Pour une matrice carré, si $P = Q$, les matrices sont semblables : elles représentent le même endomorphisme.\\
    
    Deux matrices semblables sont équivalentes, mais l'inverse n'est pas vrai.



    \chapter{Développements limités}\label{ch:developpements-limites}
        \section{Notion de développement limité}

    Une fonction admet un développement limité d'ordre $n$ en $a$ s'il existe des réels $a_0, a_1, \ldots, a_n$ tels que :\\
    \[f(x) \underset{x \rightarrow a}{=} \underbrace{a_0 + a_1 (x - a) + a_2 (x - a) + \ldots + a_n (x - a)}_\re{Partie régulière} + \underbrace{o((x - a)^n)}_\re{Reste}\]

    \twoCol {
        Pour un DL en $0$ :
        \begin{itemize}
            \item Si $f$ est paire, alors $a_{2k+1} = 0$.
            \item Si $f$ est impaire, alors $a_{2k} = 0$.
        \end{itemize}
    }{
        La partie régulière du $DL$ en 0 est unique.\\
        Le DL peut être tronqué à l'ordre $p \le n$.\\
        $f$ est équivalent en $a$ à son terme de plus petit degré.
    }
    \vspace{7pt}\\
    \textbf{DL et régularité d'une fonction :}\\
    $f$ admet un $\re{DL}_0$ en $a$ $\iff$ $f$ admet une limite finie en $a$ (continue ou prolongeable par continuité en $a$).\\
    $f$ admet un $\re{DL}_1$ en $a$ $\iff$ $f$ est dérivable en $a$ et $f'(a) = a_1$.\\
    Pou un DL d'ordre $n \ge 2$, $f$ n'est pas forcément dérivable $n$ fois en $a$.\\


\section{Formule de Taylor-Young}

    Soit $f$ définie sur un voisinage de $a$ et dérivable $n$ fois, le $\re{DL}_n$ de $f$ en $a$ s'écrit :\\
    \[f(x) \underset{x \rightarrow a}{=} \sum_{k=0}^n \frac{f^{(k)}(a)}{k!} (x - a)^k + o((x - a)^n)\]

    On peut donc identifier les coefficients $a_k = \dfrac{f^{(k)}(a)}{k!}$. Ainsi, si $f$ est $C^\infty$ au voisinage de $a$, alors elle admet un DL à tout ordre en $a$. La réciproque est fausse.\\

    \textbf{DL usuels :}
    \begin{align*}
        e^x &\underset{x \rightarrow 0}{=} 1 + x + \dfrac{x^2}{2!} + \dfrac{x^3}{3!} + \dots + \dfrac{x^n}{n!} + o(x^n) \underset{x \rightarrow 0}{=} \sum_{k=0}^n \frac{x^k}{k!} + o(x^n)\\
        \ch(x) &\re{ : tous les ordres paires de l'exponentielle}\\
        \sh(x) & \re{ : tous les ordres impaires de l'exponentielle}\\
        \cos(x) &\underset{x \rightarrow 0}{=} 1 - \dfrac{x^2}{2!} + \dfrac{x^4}{4!} - \dots + (-1)^n \dfrac{x^{2n}}{(2n)!} + o(x^{2n+1}) \quad \re{(ch() avec alternance de signe)}\\
        \sin(x) &\underset{x \rightarrow 0}{=} x - \dfrac{x^3}{3!} + \dfrac{x^5}{5!} - \dots + (-1)^n \dfrac{x^{2n+1}}{(2n+1)!} + o(x^{2n+2}) \quad \re{(sh() avec alternance de signe)}\\
        \dfrac{1}{1 - x} &\underset{x \rightarrow 0}{=} 1 + x + x^2 + \dots + x^n + o(x^n)\\
        \ln(1 + x) &\underset{x \rightarrow 0}{=} x - \dfrac{x^2}{2} + \dfrac{x^3}{3} - \dots + (-1)^{n-1} \dfrac{x^n}{n} + o(x^n)\\
        (1+x)^\alpha &\underset{x \rightarrow 0}{=} 1 + \alpha x + \dfrac{\alpha(\alpha-1)}{2!} x^2 + \dots + \dfrac{\alpha(\alpha-1)\dots(\alpha-n+1)}{n!} x^n + o(x^n)
    \end{align*}

    \newpage


\section{Opérations sur les développements limités}

    \twoCol {
        \subsection{Somme et produit}

        On peut faire la somme et le produit de deux DL$_n$.
        \smallskip

        Pour le produit, on tronque le DL à l'ordre $n$.
    }{
        \subsection{Quotient}

        Si le dénominateur a un coefficient constant non nul, on peut faire le quotient de deux DL$_n$ avec l'algorithme de division suivant les puissances croissantes.
    }

    \subsection{Changements de variable et composition de DL}

        On peut faire un changement de variable polynomial dans un DL si les deux variables tendent vers $0$.
        \medskip

        Autrement, on peut composer deux DL$_n$ en $0$ si le premier tend bien vers $0$ (coefficient constant nul).

        \twoCol {
            \subsection{Intégration de DL}

            On peut intégrer un DL$_n$ en $0$ en ajoutant une constante $F(0)$.
            \smallskip

            Toute primitive de $f$ admettra le DL intégré.
        }{
            \subsection{Dérivation de DL}

            On peut dériver un DL$_n$ en $0$ si on sait que $f'$ admet un DL$_{n-1}$ en $0$.
            \smallskip

            Cela ne prouve pas que $f'$ admet un DL$_n$ en $0$.
        }


\section{Étude locale d'une courbe}

    \subsection{Étude locale au voisinage de $x = a$}

        L'équation de la tangente en $a$ se trouve à partir du DL$_1$ de $f$ en $a$.\\

        La position de $f$ par rapport à sa tangente au voisinage de $a$ dépend du coefficient qui suit\\(pas forcément celui du degré $2$) :\\
        \begin{itemize}
            \item Si l'ordre $n$ est pair, alors le signe de $a_n (x - a)^n$ sera celui de $a_n$.\\
            Si $a_n > 0$, alors $f$ est au dessus de sa tangente, sinon elle est en dessous.
            \item Si $n$ est impair, $f$ traversera sa tangente en $a$ (on étudie le signe de $a_n (x - a)^n$).
        \end{itemize}

    \subsection{Développement limité généralisé (DLG) en $\pm \infty$ et branche infinie}

        $f$ admet un $DL$ à l'ordre $n$ en $\pm \infty$ s'il existe des réels $a_0, \dots, a_n$ tels que :\\
        \[f(x) \underset{\pm \infty}{=} a_0 + \frac{a_1}{x} + \dots + \frac{a_n}{x^n} + o\left(\frac{1}{x^n}\right)\]

        Pour calculer des DLG, en $\pm \infty$, on fait souvent le changement de variable $x = \dfrac{1}{t}$ et on calcule un DL en $0$.

        Si $f(x) \underset{\pm \infty}{=} ax + b + o(1)$, alors un DL de $\dfrac{f(x)}{x}$ en $\pm \infty$ est $a + \dfrac{b}{x} + o\left(\dfrac{1}{x}\right)$.\\

        Ainsi, la courbe de $f$ admet comme asymptote la droite $y = ax + b$ en $\pm \infty$.
        Le signe du terme suivant permet de déterminer la position locale de $f$ par rapport à son asymptote.


\end{document}
