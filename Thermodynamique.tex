\documentclass[13pt, twoside, a4paper, french]{report}

\usepackage{ficheslib}
\newcommand*{\getSubject}{Thermodynamique}

\begin{document}
    \title{\getSubject}
    \author{Clément GRENNERAT}
    \date{Mars 2023}
    \pagestyle{non-chapter-style}


    \chapter{Energie et notions fondamentales de thermodynamique}\label{ch:energie-et-notions-fondamentales-de-thermodynamique}


        \section{Rappers sur l'énergie}\label{sec:rappers-sur-l'energie}

            \subsection{Énergie primaire, énergie utile et rendement}\label{subsec:energie-primaire-energie-utile-et-rendement}

                \begin{itemize}
                    \item \underline{Énergie primaire :} énergie fournie par la nature (soleil, vent, pétrole, etc.).
                    \item \underline{Énergie finale :} énergie consommée par l'utilisateur final (convertie et transportée : électricité, essence\ldots).
                    \item \underline{Énergie utile :} énergie utile à l'action souhaitée après de nouvelles conversions.
                \end{itemize}
                Le rendement est le rapport entre l'énergie utile et l'énergie primaire.
                C'est le produit de tous les rendements des différentes conversions.

            \subsection{Chaine énergétique}\label{subsec:chaine-energetique}

                On représente en chaîne :
                \begin{itemize}
                    \item \underline{Les réservoirs :} rectangles avec le nom du réservoir et la forme d'énergie stockée.
                    \item \underline{Les convertisseurs :} ovales.
                    \item \underline{Le sens des transferts :} flèches avec la forme d'énergie transférée (et le rendement).
                \end{itemize}


        \section{Notions fondamentales}\label{sec:notions-fondamentales}

            \subsection{Système}\label{subsec:systeme}

                La thermodynamique étudie les interactions entre un système $\sigma$ et le milieu extérieur $\sigma_1$.\\
                Le système a une frontière réelle ou imaginaire.
                \vspace{5pt}
                \begin{itemize}
                    \item \underline{Système ouvert :} échange d'énergie et de matière avec $\sigma_1$
                    \item \underline{Système fermé :} échange d'énergie, mais pas de matière avec $\sigma_1$
                    \item \underline{Système isolé :} aucun échange
                \end{itemize}

            \subsection{Transformations}\label{subsec:transformations}

                Le système passe d'un état initial à un état final.
                Si l'état initial est le même que l'état final, la transformation est cyclique.
                Sinon, elle est ouverte.\\

                Le système est caractérisé par des variables d'état.
                Certaines sont extensives (dépendent de l'étendue du système), et d'autres sont intensives.
                Les variables extensives sont additives.\\

                \textbf{Transformations réalisées avec une variable constante :}
                \vspace{5pt}
                \begin{itemize}
                    \item \underline{Isochore :} $V$ constant
                    \item \underline{Isobare :} $P$ constante (Monobare si $P_i = P_f$, mais varie au cours de la transformation).
                    \item \underline{Isotherme :} $T$ constante (Monotherme si $T_i = T_f$, mais varie au cours de la transformation).
                \end{itemize}
                \vspace{5pt}
                (Une transformation monotherme réversible est isotherme.)\\

                \textbf{Types de transferts d'énergie}
                \vspace{5pt}
                \begin{itemize}
                    \item \underline{Chaleur $Q$ :} échange d'énergie microscopique (transfert d'agitation thermique).
                    \item \underline{Travail $W$ :} dû aux forces extérieures qui s'exercent sur le système.
                \end{itemize}
                \vspace{5pt}
                Si $Q = 0$, la transformation est adiabatique.\\


    \chapter{Etat de la matière et gaz parfait}\label{ch:etat-de-la-matiere-et-gaz-parfait}


        \section{Les propriétés des gaz}\label{sec:les-proprietes-des-gaz}

            \textbf{Loi des gaz parfaits :} $P V = n R T$


        \section{Capacité calorifique d’un système et coefficient de Laplace d’un gaz}\label{sec:capacite-calorifique-dun-systeme-et-coefficient-de-laplace-dun-gaz}

            La capacité thermique est une grandeur extensive en $J.K^{-1}$.\\
            C'est l'énergie à apporter pour élever la température d'un degré.
            \vspace{7pt}\\
            On définit de même des grandeurs intensives associés telles que la capacité thermique massique $c_m$ et la capacité thermique molaire $\overline{C}$.\\

            \textbf{Pour les gaz}, comme ils sont compressibles, on distingue deux modes de chauffage :
            \vspace{3pt}
            \begin{itemize}
                \item \underline{Chauffage isobare :} $P$ constante : on définis $C_p$ (capacité calorifique du gaz à pression constante).
                \item \underline{Chauffage isobare :} $V$ constante : on définis $C_v$ (capacité calorifique du gaz à volume constant).
            \end{itemize}
            \vspace{5pt}
            Le \textbf{coefficient de Laplace} d'un gaz vaut : $\gamma = \dfrac{C_p}{C_v}$.\\
            En première approximation, pour les gaz monoatomiques, $\gamma \approx 1,67$ et pour les gaz diatomiques, $\gamma \approx 1,4$.


        \section{Les changements d’état d’un corps pur}\label{sec:les-changements-detat-dun-corps-pur}

            \twoCol[41] {
                Un changement d'état se fait sous température et pression constantes.\\
                \vspace{0pt}
                \resizebox{\textwidth}{!}{
                    \includestandalone[mode=buildnew,page=1, scale]{out/phase_transitions_1}
                }
            }{

                \subsection{Échange de chaleur au cours d’un changement d’état}\label{subsec:echange-de-chaleur-au-cours-dun-changement-detat}

                \begin{itemize}
                    \item Pour la fusion, sublimation et l'évaporation, il faut apporter de l'énergie au système (chaleur positive, endothermique).
                    \item Pour la solidification, condensation et liquéfaction, la chaleur sera négative (exothermique).
                \end{itemize}
                \vspace{5pt}
                On définit ainsi la chaleur latente de fusion $L_{fus}$, de sublimation $L_{sub}$, d'évaporation $L_{vap}$ (et les autres de signe opposé).\\
            }
            \vspace{-7pt}

            \subsection{Représentations graphiques des changements d’état – diagramme de phase}\label{subsec:representations-graphiques-des-changements-detat--diagramme-de-phase}

                \resizebox{.35\textwidth}{!}{
                \includestandalone[mode=buildnew,page=1, scale]{out/phase_transitions_2}
            }
                \resizebox{.5\textwidth}{!}{
                \includegraphics[width=\textwidth]{graphs/Clapeyron}
            }

        \vspace{-7pt}
        La pression de vapeur saturante $P_{vs}$ est la pression à laquelle le liquide et la vapeur sont en équilibre.\\
        $P_{vs}$ croit avec la température.

        \section{Mélange de corps pur}\label{sec:melange-de-corps-pur}

            \twoCol {
                Fraction molaire : $x_i = \dfrac{n_i}{x_{tot}}$
            }{
                Fraction massique : $w_i = \dfrac{m_i}{m_{tot}}$
            }
            \vspace{5pt}\\
            La masse molaire du mélange correspond à la moyenne pondérée des masses polaires des corps purs.\\
            On définis de même la pression partielle $P_i = x_i \times P$ où la loi des gaz parfaits s'applique aussi.

\end{document}
