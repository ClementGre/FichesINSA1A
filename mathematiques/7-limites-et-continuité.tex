\section{Bornes sup/inf d'une fonction}\label{sec:bornes-sup/inf-d'une-fonction}

$f$ est majorée sur $D$ si $\exists \alpha \in \mathbb{R}\ /\ \forall x \in D, f(x) \le \alpha$ (resp. minorée)\\

Ainsi, $f$ admet un \underline{plus petit majorant} et un \underline{plus grand minorant} : borne supérieure et borne inférieure\\
Si $f$ est non majorée, $\sup f(x) = +\infty$ (resp. non minorée, $\inf f(x) = -\infty$)\\


\section{Limites d'une fonction}\label{sec:limites-d'une-fonction}

Une propriété esr vraie au voisinage d'un réel $x_0$ s'il existe un intervalle de la forme $I = ]x_0 - \alpha ; x_0 + \alpha[$ où la propriété est vraie sur $I \textbackslash \{x_0\}$.
De même pour un voisinnage en $\pm \infty$, un intervalle $]A ; +\infty[$ ou $]-\infty ; A[$.


\twoCol {

    \subsection{Limite finie en l'infini}\label{subsec:limite-finite-en-l'infini}
    $\forall \epsilon > 0,\ \exists X_\epsilon \in \mathbb{R}\ / \ \forall x \in D_f,\ x > X_\epsilon \Rightarrow |f(x) - l| < \epsilon$\\
    On a alors une asymptote horizontale $y = l$ à $C_f$.\\

    \subsection{Limite infinie en l'infini}\label{subsec:limite-infinie-en-l'infini}
    $\forall A \in \mathbb{R},\ \exists X_A \in \mathbb{R}\ / \ \forall x \in D_f,\ x > X_A \Rightarrow f(x) > A$\\
}{
    \subsection{Limite infinie en un réel}\label{subsec:limite-infinie-en-un-reel}
    $\forall A \in \mathbb{R}, \exists \alpha_A > 0,\ \forall x \in D_f,\ |x - x_0| < \alpha_A \Rightarrow f(x) > A$\\
    On a alors une asymptote verticale $x = x_0$ à $C_f$.\\

    \subsection{Limite finie en un réel}\label{subsec:limite-finite-en-un-reel}
    $\forall \epsilon > 0,\ \exists \alpha_\epsilon > 0,\ \forall x \in D_f,\ |x - x_0| < \alpha_\epsilon \Rightarrow |f(x) - l| < \epsilon$\\
}
\subsection{Limite en général}\label{subsec:limite-en-general}

Dans le cas général, $f$ admet $l$ pour limite en $x_0$ si pour tout voisinage $V_l$ de $l$, il existe un voisinage $V_x_0$ de $x_0$ tel que $f(V_x_0 \cap D_f) \subset V_l$\\
Une limite est unique, démontre la continuité ($l = f(x_0)$), et si elle est finie, $f$ est bornée au voisinage de $x_0$.\\

\subsection{Limites à gauche/droite}\label{subsec:limites-a-gauche/droite}

Si $x_0 \notin D_f$, $f$ admet une limite en $x_0$ ssi $f$ admet une limite à gauche et à droite identique en $x_0$.\\
Si $x_0 \in D_f$, il faut que ces limites vaillent $l = f(x_0)$. C'est la continuité.\\





