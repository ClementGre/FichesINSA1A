\section{Logique}\label{sec:logique}
  
  \subsection{Connecteurs logiques $\neg$, $\land$, $\lor$, ($\oplus$ ou $\veebar$)}\label{subsec:connecteurs-logiques}
    
    \fourCol[33][18][19]{
      
      \textbf{La négation $\neg$}, pour les ensembles, correspond au complémentaire du sous ensemble $A$ dans $E$, noté $\complement_{E}A$, $\overline{A}$ ou $A^C$ \\
      Négation $\neq$ Contraire (Non jeune $\neq$ vieux)
    }{
      \textbf{La conjonction $\land$} pour ET, correspond à l'intersection $\cap$.
    }{
      \textbf{La disjonction $\lor$} OU, correspond à l'union $\cup$.
    }{
      \textbf{La disjonction exclusive $\oplus$ ou $\veebar$} XOR, l'une des deux vrais et l'autre nécessairement fausse.\\
      Correspond à la différence symétrique $\Delta$.
    }
    
    \vspace*{7pt}
    
    \twoCol{
      \makebox[3cm]{\textbf{Règles de Morgan :}} $\neg(P \lor Q) \Leftrightarrow (\neg P) \land (\neg Q)$ \\
      \makebox[3cm]{} $\neg(P \land Q) \Leftrightarrow (\neg P) \lor (\neg Q)$
    }{
      \makebox[3cm]{\textbf{Distributivité :}} $(P \lor Q) \land R \Leftrightarrow (P \land R) \lor (Q \land R)$ \\
      \makebox[3cm]{} $(P \land Q) \lor R \Leftrightarrow (P \lor R) \land (Q \lor R)$
    }
    
    \makebox[3cm]{\textbf{Produit cartésien :}} $A \times B = \{(a, b),\ a \in A\ et\ b \in B\}$
  
  \subsection{Connecteurs logiques $\Rightarrow$ et $\Leftrightarrow$}\label{subsec:connecteurs-logiques-2}
    
    \fourCol[35][19][27]{
      \textbf{L'implication} de P à Q n'est fausse que lorsque P est vraie et Q fausse.\\
      \makebox[2.3cm][r]{\textbf{Correspond à :}} $\neg P \lor Q$\\
      \makebox[2.3cm][r]{\textbf{Négation :}} $P \land \neg Q$\\
      \makebox[2.3cm][r]{\textbf{Réciproque :}} $Q \Rightarrow P$\\
      \makebox[2.3cm][r]{\textbf{Contraposée :}} $(\neg Q) \Rightarrow (\neg P)$\\\;(même valeur de vérité).
    }{
      \begin{tabular}[t]{|c|c|c|}
        \hline
        \headRow
        P                     & Q                     & $P \Rightarrow Q$     \\ \hline
        \cellcolor{green!15}V & \cellcolor{green!15}V & \cellcolor{green!15}V \\ \hline
        \cellcolor{green!15}V & \cellcolor{red!15}F   & \cellcolor{red!15}F   \\ \hline
        \cellcolor{red!15}F   & \cellcolor{green!15}V & \cellcolor{green!15}V \\ \hline
        \cellcolor{red!15}F   & \cellcolor{red!15}F   & \cellcolor{green!15}V \\ \hline
      \end{tabular}%
    }{
      \textbf{L'équivalence} de P et Q n'est vraie que si P et Q ont même valeur de vérité.\\
      L'équivalence a la même valeur de vérité que la double implication (nécessaire et suffisant).
    }{
      \begin{tabular}[t]{|c|c|c|}
        \hline
        \headRow
        P                     & Q                     & $P \Leftrightarrow Q$ \\ \hline
        \cellcolor{green!15}V & \cellcolor{green!15}V & \cellcolor{green!15}V \\ \hline
        \cellcolor{green!15}V & \cellcolor{red!15}F   & \cellcolor{red!15}F   \\ \hline
        \cellcolor{red!15}F   & \cellcolor{green!15}V & \cellcolor{red!15}F   \\ \hline
        \cellcolor{red!15}F   & \cellcolor{red!15}F   & \cellcolor{green!15}V \\ \hline
      \end{tabular}
    }
  
  \subsection{Quantificateurs}\label{subsec:quantificateurs}
    
    $\forall x \in A, P(x)$ : quel que soit $x$ élément de $A$ (ou pour tout $x$ appartenant à $A$). \hfill \textbf{Négation :} $\exists x \in A, \neg P(x)$\\
    $\exists x \in A, P(x)$ : il existe au moins un élément $x$ de $A$ vérifiant $P$. \hfill \textbf{Négation :} $\forall x \in A, \neg P(x)$\\
    $\exists !x \in A, P(x)$ : il y a existence et \textbf{unicité} de l'élément $x$ dans $A$ vérifiant $P$.


\section{Méthodes de raisonnement}\label{sec:methodes-de-raisonnement}
  
  \begin{multicols}{2}
    \raggedcolumns
    
    \subsection{Raisonnement direct}\label{subsec:raisonnement-direct}
      
      (Par déduction ou hypothèse auxiliaire).\\
      Hypothèse $\Rightarrow$ Conclusion.
    
    \subsection{Raisonnement par disjonction des cas}\label{subsec:raisonnement-par-disjonction-des-cas}
      
      Raisonnement direct dans lequel l'hypothèse peut se décomposer en plusieurs sous hypothèses :\\
      $H \Leftrightarrow H_1 \land H_2$\\
      On montre ensuite $H_1 \Rightarrow C$ et $H_2 \Rightarrow C$
    
    \subsection{Raisonnement par contraposition}\label{subsec:raisonnement-par-contraposition}
      
      $(H \Rightarrow C) \Leftrightarrow (\neg C \Rightarrow \neg H)$\\
      \textit{Exemple :} La contraposée de \og $\forall n \in \mathbb{Z}$, si $n$ est pair, alors $n^2$ est pair. \fg\ est \og Si $n^2$ est impair, alors $n$ est impair \fg
    
    \subsection{Raisonnement par l'absurde}\label{subsec:raisonnement-par-l'absurde}
      
      On suppose le contraire et on montre que cela vient contredire une proposition vraie.\\
      \textit{Exemple :} Démontrons que $\sqrt{2}$ est irrationnel $\rightarrow$ supposons que $\sqrt{2}$ est rationnel.
      On aurait donc $2 = \frac{p^2}{q^2}$ avec $(p, q) \in \mathbb{N}^2$ et $p$ et $q$ premiers entre eux.
      Donc $p^2 = 2q^2$ alors $p$ pair, $p = 2p'$ d'où $q^2 = 2p'^2$ donc $q$ serait pair.
      Ainsi, p et q ne seraient pas premiers entre eux, on a bien une contradiction.
    
    \subsection{Raisonnement par contre-exemple}\label{subsec:raisonnement-par-contre-exemple}
      
      Il suffit de trouver un contre exemple pour prouver qu'une propriété est fausse.
    
    \subsection{Raisonnement par récurrence simple}\label{subsec:raisonnement-par-recurrence}
      
      - On montre que $P(0)$ est vraie (propriété initialisée ou fondée)\\
      - On suppose $P(n)$ et on montre $P(n+1)$ (hérédité)
    
    \subsection{Raisonnement par analyse-synthèse}\label{subsec:raisonnement-par-analyse-synthese}
      
      - Analyse : on établit une liste de potentielles solutions parmi lesquelles toutes les solutions réelles sont nécessairement incluses.\\
      - Synthèse : pour chacune de ces solutions,  on détermine si elles sont viables ou non.
  
  \end{multicols}
  
  
