\section{Matrices et opérations}

    \subsection{Définition}

        Une matrice, notée $A \in \mathcal{M}_{n,p}$ est un tableau de scalaires à $n$ lignes et $p$ colonnes.\\
        Ses coefficients sont notés $a_{ij}$, $i$ étant la ligne et $j$ la colonne.\\

        \twoCol {
            \underline{Matrices particulières :}
            \begin{itemize}
                \item Matrice nulle : $0_\mathcal{M}_{n,p}$
                \item Matrice ligne : $n = 1$
                \item Matrice colonne : $p = 1$
                \item Matrice carrée : $n = p$, on note $A \in \mathcal{M}_{n}$
            \end{itemize}
        }{
            \underline{Matrices particulières carrées :}
            \begin{itemize}
                \item Triangulaire supérieure : $\forall i > j, a_{i,j} = 0$
                \item Triangulaire inférieure : $\forall i < j, a_{i,j} = 0$
                \item Diagonale : $\forall i \new j, a_{i,j} = 0$
                \item Identité : diagonale, avec coefs. égaux à 1.
            \end{itemize}
        }

    \subsection{Opérations sur les matrices}

        \twoCol {

            \underline{Somme et multiplication par un scalaire}

            \begin{itemize}
                \item \textbf{Loi interne $+$ :}\\Les coefficient $i,j$ s'ajoutent entre eux \\(matrices de même taille).\vspace{5pt}
                \item \textbf{Loi externe $\cdot$ :}\\Chaque coefficient est multiplié par le scalaire.
            \end{itemize}
        }{
            \underline{Produit de matrices}\\
            Soit $A \in \mathcal{M}_{n,p}$ et $B \in \mathcal{M}_{p,q}$, alors $AB = C \in \mathcal{M}_{n,q}$ :

            \[ c_{ij} = \sum_{k=1}^{p} a_{ik} b_{kj} \]
            On multiplie les lignes de $A$ par les colonnes de $B$.
        }

    \subsection{Règles de calcul}

        \underline{Puissances de matrices :}
        \begin{itemize}
            \item $A^k$ est définis uniquement pour $A \in \mathcal{M}_{n}$ et vaut $A \times A \times \dots \times A$ ($k$ fois).\vspace{5pt}
        \end{itemize}

        \underline{Produit matriciel :}
        \begin{itemize}
            \item On distingue la distributivité à gauche et à droite car $\mathcal{M}_{n,p}(\mathbb{K})$ n'est pas commutatif.
            \item La multiplication avec un scalaire est commutative : $\lambda \times A = A \times \lambda$.
            \item L'associativité est vérifiée : $A(B C) = (A B) C = ABC$.\vspace{5pt}
        \end{itemize}

        \underline{Matrices identité :}\\
        $I_n$ est l'élément neutre pour le produit matriciel.\vspace{5pt}

        \underline{Matrices particulières :}
        \begin{itemize}
            \item Le produit de deux matrices triangulaires supérieures est triangulaire supérieur
            \item Le produit de deux matrices triangulaires inférieures est triangulaire inférieur
            \item Le produit de deux matrices diagonales est diagonal\vspace{5pt}.
        \end{itemize}

        \underline{Binôme de Newton :}\\
        Si A et B commutent ($AB = BA$) :
        \[ (A + B)^n = \sum_{k=0}^{n} \binom{n}{k} A^{n-k} B^k \]


\section{Lien avec les applications linéaires}

    \subsection{Matrice d'une application linéaire}

        Une application linéaire $f$ est entièrement déterminée par la donnée de $f(e_1)$, \ldots, $f(e_p)$.
        On peut donc représenter une application linéaire par une matrice.\\

        La matrice de $f$ relativement aux bases $\mathcal{B}$ et $\mathcal{B}'$ a les colonnes formées des coordonnées des images des vecteurs de la base de départ $\mathcal{B}$ exprimés dans la base d'arrivée $\mathcal{B}'$.\\
        On note cette matrice $[f]_{\mathcal{B},\mathcal{B}'}$.\\

        Si $f$ est un endomorphisme, on peut choisir la même base \mathcal{B} de départ et d'arrivée et on note $[f]_{\mathcal{B}}$.

    \subsection{Image d'un vecteur par une application linéaire}

        On multiplie la matrice d'une application linéaire par
        $X = \begin{pmatrix}
                 x_1\\ \vdots\\ x_n
        \end{pmatrix}$
        pour obtenir le vecteur image de $X$ par $f$.\\
        On note $Y = AX$.


        \twoCol {

            \subsection{L'espace vectoriel $\mathcal{M}_{n,p}(\mathbb{K})$}

            $\mathcal{M}_{n,p}(\mathbb{K})$ est un $\mathbb{K}$-ev. donc :
            \begin{itemize}
                \item $[f + g]_{\mathcal{B},\mathcal{B}'} = [f]_{\mathcal{B},\mathcal{B}'} + [g]_{\mathcal{B},\mathcal{B}'}$
                \item $[\lambda f]_{\mathcal{B},\mathcal{B}'} = \lambda [f]_{\mathcal{B},\mathcal{B}'}$
            \end{itemize}
            \vspace{5pt}\\

            On définis ainsi un isomorphisme :\vspace{-8pt}\\
            \[\begin{cases}
                  \mathcal{L}(E,\ F) &\to \mathcal{M}_{n,p}(\mathbb{K})\\
                  f &\mapsto [f]_{\mathcal{B},\mathcal{B}'}
            \end{cases}\]\vspace{-8pt}\\—}
            Où $\dim(\mathcal{M}_{n,p}(\mathbb{K})) = np = \dim(F) \times \dim(E)$
        }{
            \subsection{Composées d'applications linéaires et produit matriciel}

            Soient $f\ :\ E \rightarrow F$ et $g\ :\ F \rightarrow G$ deux applications linéaires :

            \[[g \circ f]_{\mathcal{B}_E,\mathcal{B}_G} = [g]_{\mathcal{B}_F,\mathcal{B}_G} \times [f]_{\mathcal{B}_E,\mathcal{B}_F}\]

            On met en première la matrice de l'application appliquée en dernière.
        }


\section{Inversion de matrices}

    $A \in \mathcal{M}_n$ est inversible $\iff \exists A^{-1} \in \mathcal{M}_n\ /\ A^{-1}A = AA^{-1} = I_n$.\\
    Cela équivaut à dire que $A$ est la matrice d'une application linéaire bijective.\\

    Ainsi, on a $(A^{-1})^{-1} = A$ et $(AB)^{-1} = B^{-1} A^{-1}$.\\

    \underline{Calcul de l'inverse :} \vspace{4pt}\\
    Soit $A = [f]_\mathcal{B}$, alors $A^{-1} = [f^{-1}]_\mathcal{B}$.
    (L'écriture sous forme d'une AL correspond à celle matricielle.) \vspace{2pt}\\
    On résous donc le système $Y = AX$ : on exprime $X$ en fonction de $Y$ quelconque et on obtient $A^{-1}$.

    \twoCol{
        \section{Rang d'une matrice}

        Le rang d'une matrice $A$ est le rang des vecteurs colonne de $A$, et le rang de l'AL associée.\\

        Pour une matrice carré :\\ $A \in \mathcal{M}_n$ est inversible $\iff \rg(A) = n$.\\
        Pour une matrice rectangulaire :\\ $A \in \mathcal{M}_{n,p}$, $\rg(A) \le \min(n,p)$.
    }{
        \section{Transposition}

    }
    \vspace{1pt}\\


\section{Matrices de changements de bases}

    Matrice de passage de $\mathcal{B}$ à $\mathcal{B}'$ : $\re{Pass}_{\mathcal{B} \rightarrow \mathcal{B}'} = [\mathcal{B}']_\mathcal{B} =  [id_E]_{\mathcal{B}',\mathcal{B}}$.\\
    Elle exprime en colonne les coordonnés des vecteurs de $\mathcal{B}'$ exprimés dans la base $\mathcal{B}$.\\

    En multipliant par $[\mathcal{B}']_\mathcal{B}$ on transforme les coordonnés exprimés dans $\mathcal{B}'$ en coordonnés exprimés dans $\mathcal{B}$.\\
    Comme pour la composition de fonctions, on lit de droite à gauche : $[\mathcal{B''}]_{\mathcal{B}} = [\mathcal{B'}]_{\mathcal{B}} \times [\mathcal{B}'']_{\mathcal{B}'}$\\
    Pour un vecteur $x$ : $[x]_{\mathcal{B}} = [\mathcal{B'}]_\mathcal{B} \times [x]_{\mathcal{B}'}$
    \vspace{5pt}\\
    On a $([\mathcal{B}']_\mathcal{B})^{-1} = [\mathcal{B}]_{\mathcal{B}'}$ : il faudra souvent inverser la matrice de passage.
    \vspace{5pt}\\
    Pour changer la base de départ d'une AL, on multiplie à droite par la matrice de passage, et pour changer la base d'arrivée, on multiplie à gauche : $[f]_{\mathcal{B}',\mathcal{C}'} = [\mathcal{C}]_{\mathcal{C'}} \times [f]_{\mathcal{B},\mathcal{C}} \times [\mathcal{B'}]_{\mathcal{B}}$\\

    $A$ et $B$ sont équivalentes$\iff$sont la même AL dans des bases différentes$\iff A = Q^{-1} B P \iff \boldsymbol{\rg(A) = \rg(B)}$.\\
    Pour une matrice carré, si $P = Q$, les matrices sont semblables : elles représentent le même endomorphisme.\\

    Deux matrices semblables sont équivalentes, mais l'inverse n'est pas vrai.

