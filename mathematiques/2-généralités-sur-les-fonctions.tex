\section{Généralités}\label{sec:generalites}
  
  \subsection{Différence entre applications et fonctions}\label{subsec:difference-entre-applications-et-fonctions}

    Une application est une relation entre deux ensembles.\\
    Une fonction est une application d'une partie $D_f$ d'un ensemble de départ. $D_f$ est appelé ensemble de définition.\\
    Une fonction n'est donc pas forcément définie sur l'entièreté de l'ensemble de départ.
    
    \vspace*{5pt}
    
    \fourCol[25][26][20]{
      \subsection{Image directe}\label{subsec:image-directe}
      \textbf{Ensemble des images} de $A$ par $f$ :\\ $f(A) = \{f(x)\ /\ x \in A\}$
    }{
      \subsection{Image réciproque}\label{subsec:image-reciproque}
      
      \textbf{Ensemble des antécédents} d'un ensemble.\\
      C'est l'image directe de l'ensemble par $f^{-1}$.
    }{
      \subsection{Restriction}\label{subsec:restriction-d'une-application}
      
      On note $f_{|A}(x) = f(x)$ pour tout $x \in A$, avec $A \subset D_f$.
    }{
      
      \subsection{Composition}\label{subsec:composition}
      $f \circ g(x) = f(g(x))$\\
      \textbf{Associative} mais non commutative :\\
      $(f \circ g) \circ h = f \circ (g \circ h)$
    }
    
    \vspace*{5pt}
    
    \twoCol[55]{
      \subsection{Injectivité}\label{subsec:injectivite}
      
      \textbf{Au plus} un antécédent.\\
      Fonction injective : $f(x_1) = f(x_2) \Rightarrow x_1 = x_2$.\\
      Non injective : $\exists (x_1, x_2) \in D_f^2\ /\ f(x_1) = f(x_2) \land x_1 \neq x_2$
    }{
      \subsection{Surjectivité}\label{subsec:surjectivite}
      
      \textbf{Au moins} un antécédent.\\
      Fonction surjective : $\forall y \in F,\ \exists x \in D_f\ /\ y = f(x)$.\\
      Non surjective : $\exists y \in F\ /\ \forall x \in D_f,\ y \neq f(x)$
    }
  
  \subsection{Bijectivité ou Réciprocité}\label{subsec:bijectivite-ou-reciprocite}
    À la fois injective et surjective : $\forall y \in F,\ \exists! x \in D_f\ /\ y = f(x)$.\\
    On note alors $x = f^{-1}(y)$ la bijection réciproque de f.


\section{Fonctions de $\mathbb{R}$ dans $\mathbb{R}$}\label{sec:fonctions-de-r-dans-r}
  
  \twoCol{
    
    \subsection{Sens de variation}\label{subsec:sens-de-variation}
    
    $f \circ g$ est :
    \begin{itemize}
      \item Croissante si $f$ et $g$ sont de même monotonie.
      \item Décroissante si $f$ et $g$ sont de sens de monotonies contraires.
    \end{itemize}
    
    \vspace*{7pt}
    Si $f$ est strictement monotone, alors elle est injective (au plus un antécédent par image).
  }{
    
    \subsection{Majorant et minorant}\label{subsec:majorant-et-minorant}
    
    \begin{outline}
      \1 \textbf{$\alpha$ maximum de A} si $\alpha$ est à la fois majorant et élément de A : $\alpha = \max(A)$ \;\; (resp. $\min(A)$).
      \1 \textbf{$\alpha$ borne supérieure de A} si $\alpha$ est le plus petit des majorants : $\alpha = \sup(A)$ \;\; (resp. $\inf(A)$).
      \2 si A est non vide non majoré : $\sup(A) = +\infty$
      \2 si A est non vide non minoré : $\inf(A) = -\infty$
    \end{outline}
    Si le maximum existe, il est égal à la borne supérieure.
  }
  
  \subsection{Parité}\label{subsec:parite}
    
    \begin{itemize}
      \item Si $f$ et $g$ sont paires, $f + g$ est paire (resp. impaires).
      \item Si $f$ et $g$ ont même parité, $f \times g$ est paire (resp. $\sfrac{f}{g}$).
      \item Si $f$ et $g$ ont des parités contraires, $f \times g$ est impaire (resp. $\sfrac{f}{g}$).
      \item Si $f$ est paire, $g \circ f$ est paire.
      \item Si $f$ est impaire, $g \circ f$ a la même parité que $g$.
    \end{itemize}
    
    \vspace*{7pt}
    
    \twoCol {
      
      \subsection{Périodicité}\label{subsec:periodicite}
      
      $T$-périodique si $\forall (x,x+T) \in D_f^2$,\; $f(x + T) = f(x)$.
      \begin{itemize}
        \item Si $f$ et $g$ sont $T$-périodiques,\\$f + g$ et $f \times g$ sont $T$-périodiques.
        \item Si $f$ est $T$-périodique,\\$g \circ f$ est $T$-périodique.
      \end{itemize}
    }{
      
      \subsection{Bijectivité et symétrie}\label{subsec:bijectivite-et-symetrie}
      
      \begin{itemize}
        \item Si $f$ est une bijection, $C_f$ et $C_{f^{-1}}$ sont symétriques par rapport à $y = x$.
        \item $x \mapsto f(-x)$ : symétrie par l'axe des ordonnées.
        \item $x \mapsto -f(x)$ : symétrie par l'axe des abscisses.
        \item $x \mapsto f(x + a)$ : translation de vecteur $-a \vec{\imath}$.
        \item $x \mapsto f(x) + a$ : translation de vecteur $a \vec{\jmath}$.
        \item $x \mapsto f(ax)$ : réduction/agrandissement sur axe x.
        \item $x \mapsto af(x)$ : agrandissement/réduction sur axe y.
      \end{itemize}
    }


