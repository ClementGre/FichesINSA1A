\section{Bornes sup/inf d'une fonction}\label{sec:bornes-sup/inf-d'une-fonction}

$f$ est majorée sur $D$ si $\exists \alpha \in \mathbb{R}\ /\ \forall x \in D, f(x) \le \alpha$ (resp. minorée)\\

Ainsi, $f$ admet un \underline{plus petit majorant} et un \underline{plus grand minorant} : borne supérieure et borne inférieure\\
Si $f$ est non majorée, $\sup f(x) = +\infty$ (resp. non minorée, $\inf f(x) = -\infty$)


\section{Limites d'une fonction}\label{sec:limites-d'une-fonction}

Une propriété esr vraie au voisinage d'un réel $x_0$ s'il existe un intervalle de la forme $I = ]x_0 - \alpha ; x_0 + \alpha[$ où la propriété est vraie sur $I \textbackslash \{x_0\}$.
De même pour un voisinnage en $\pm \infty$, un intervalle $]A ; +\infty[$ ou $]-\infty ; A[$.


\twoCol {

    \subsection{Limite finie en l'infini}\label{subsec:limite-finite-en-l'infini}
    $\forall \epsilon > 0,\ \exists X_\epsilon \in \mathbb{R}\ / \ \forall x \in D_f,\ x > X_\epsilon \Rightarrow |f(x) - l| < \epsilon$\\
    On a alors une asymptote horizontale $y = l$ à $C_f$.\\

    \subsection{Limite infinie en l'infini}\label{subsec:limite-infinie-en-l'infini}
    $\forall A \in \mathbb{R},\ \exists X_A \in \mathbb{R}\ / \ \forall x \in D_f,\ x > X_A \Rightarrow f(x) > A$\\
}{

    \subsection{Limite infinie en un réel}\label{subsec:limite-infinie-en-un-reel}
    $\forall A \in \mathbb{R}, \exists \alpha_A > 0,\ \forall x \in D_f,\ |x - x_0| < \alpha_A \Rightarrow f(x) > A$\\
    On a alors une asymptote verticale $x = x_0$ à $C_f$.\\

    \subsection{Limite finie en un réel}\label{subsec:limite-finite-en-un-reel}
    $\forall \epsilon > 0,\ \exists \alpha_\epsilon > 0,\ \forall x \in D_f,\ |x - x_0| < \alpha_\epsilon \Rightarrow |f(x) - l| < \epsilon$\\
}

\twoCol {
    \subsection{Limite en général}\label{subsec:limite-en-general}

    Dans le cas général, $f$ admet $l$ pour limite en $x_0$ si pour tout voisinage $V_l$ de $l$, il existe un voisinage $V_{x_0}$ de $x_0$ tel que $f(V_{x_0} \cap D_f) \subset V_l$.
    Une limite est unique, démontre la continuité ($l = f(x_0)$), et si elle est finie, $f$ est bornée au voisinage de $x_0$.\\
}{
    \subsection{Limites à gauche/droite}\label{subsec:limites-a-gauche/droite}

    Si $x_0 \notin D_f$, $f$ admet une limite en $x_0$ ssi $f$ admet une limite à gauche et à droite identique en $x_0$.\\
    Si $x_0 \in D_f$, il faut que ces limites vaillent $l = f(x_0)$.
    C'est la continuité.\\
}

\twoCol {
    \subsection{Opérations sur les limites}\label{subsec:operations-sur-les-limites}

    \underline{Formes indéterminées :} $+\infty -\infty$, $0 \times \infty$, $\dfrac{\infty}{\infty}$, $\dfrac{0}{0}$, $\infty^0$, $0^0$, $1^{\infty}$.\\
    \underline{Polynômes et fractions rationnelles :} termes de plus haut degré.\\

    $f$ et $g$ équivalentes : $f \underset{x_0}{\sim} g \iff \lim_{x_0} \dfrac{f}{g} = 1$\\
    $f$ négligeable devant $g$ : $f \underset{x_0}{=} o(g) \iff \lim_{x_0} \dfrac{f}{g} = 0$\\

    \subsection{Branches infinies}\label{subsec:branches-infinies}

    Si $\limm{x}{+\infty} (f(x) - (ax+b)) = 0$ \\
    $y = ax + b$ est asymptote oblique à $C_f$ en $+\infty$.\\

    Si $\limm{x}{+\infty} \dfrac{f(x)}{x} = a$ et $\limm{x}{+\infty} (f(x) -ax) = \pm \infty$ :\\
    $f$ admet une branche parabolique de direction asymptotique à la droite $y = ax$ en $+\infty$.\\
    Si $a = \pm \infty$, la direction asymptotique est l'axe des ordonnées.\\
}{

    \subsection{Ordre et limites}\label{subsec:ordre-et-limites}
    Si $f$ croissante (resp. décroissante) sur $I =\ ]a, b[$\\
    où $a, b \in [-\infty, +\infty]$ :
    \begin{itemize}
        \item Si $f$ majorée sur $I$, elle admet une limite à gauche finie en $b$.
        \item Si $f$ non majorée sur $I$, $\limm{x}{b^-} f(x) = +\infty$.
        \item Dans les deux cas $\limm{x}{b^-} f(x) = \sup f(x)$.
    \end{itemize}
    \phantom{s}\\
    \underline{Croissance comparées :}
    \begin{itemize}
        \item Les puissance du logarithme sont négligeables devant les fonctions puissances positives.
        \item Les fonctions puissance sont négligeables devant les puissances positives de l'exponentielle.
    \end{itemize}
    \phantom{s}\\
    \underline{Ordre et limites :}
    Si $f \le g$, $\limm{x}{x_0} f(x) \le \limm{x}{x_0} g(x)$.\\
    On peut étendre ceci pour définir le théorème de l'encadrement.
}


\section{Continuité d'une fonction}\label{sec:continuite-d'une-fonction}


\section{Comparaison locale de deux fonctions}\label{sec:comparaison-locale-de-deux-fonctions}






