\section{Bornes sup/inf d'une fonction}\label{sec:bornes-sup/inf-d'une-fonction}

$f$ est majorée sur $D$ si $\exists \alpha \in \mathbb{R}\ /\ \forall x \in D, f(x) \le \alpha$ (resp. minorée)\\

Ainsi, $f$ admet un \underline{plus petit majorant} et un \underline{plus grand minorant} : borne supérieure et borne inférieure\\
Si $f$ est non majorée, $\sup f(x) = +\infty$ (resp. non minorée, $\inf f(x) = -\infty$)


\section{Limites d'une fonction}\label{sec:limites-d'une-fonction}

Une propriété est vraie au voisinage d'un réel $x_0$ s'il existe un intervalle de la forme $I = ]x_0 - \alpha ; x_0 + \alpha[$ où la propriété est vraie sur $I \textbackslash \{x_0\}$.
De même pour un voisinnage en $\pm \infty$, un intervalle $]A ; +\infty[$ ou $]-\infty ; A[$.


\twoCol {

    \subsection{Limite finie en l'infini}\label{subsec:limite-finite-en-l'infini}
    $\forall \epsilon > 0,\ \exists X_\epsilon \in \mathbb{R}\ / \ \forall x \in D_f,\ x > X_\epsilon \Rightarrow |f(x) - l| < \epsilon$\\
    On a alors une asymptote horizontale $y = l$ à $C_f$.\\

    \subsection{Limite infinie en l'infini}\label{subsec:limite-infinie-en-l'infini}
    $\forall A \in \mathbb{R},\ \exists X_A \in \mathbb{R}\ / \ \forall x \in D_f,\ x > X_A \Rightarrow f(x) > A$\\
}{

    \subsection{Limite infinie en un réel}\label{subsec:limite-infinie-en-un-reel}
    $\forall A \in \mathbb{R}, \exists \alpha_A > 0,\ \forall x \in D_f,\ |x - x_0| < \alpha_A \Rightarrow f(x) > A$\\
    On a alors une asymptote verticale $x = x_0$ à $C_f$.\\

    \subsection{Limite finie en un réel}\label{subsec:limite-finite-en-un-reel}
    $\forall \epsilon > 0,\ \exists \alpha_\epsilon > 0,\ \forall x \in D_f,\ |x - x_0| < \alpha_\epsilon \Rightarrow |f(x) - l| < \epsilon$\\
}

\twoCol {
    \subsection{Limite en général}\label{subsec:limite-en-general}

    Dans le cas général, $f$ admet $l$ pour limite en $x_0$ si pour tout voisinage $V_l$ de $l$, il existe un voisinage $V_{x_0}$ de $x_0$ tel que $f(V_{x_0} \cap D_f) \subset V_l$.
    Une limite est unique, démontre la continuité ($l = f(x_0)$), et si elle est finie, $f$ est bornée au voisinage de $x_0$.\\
}{
    \subsection{Limites à gauche/droite}\label{subsec:limites-a-gauche/droite}

    Si $x_0 \notin D_f$, $f$ admet une limite en $x_0$ ssi $f$ admet une limite à gauche et à droite identique en $x_0$.\\
    Si $x_0 \in D_f$, il faut que ces limites vaillent $l = f(x_0)$.
    C'est la continuité.\\
}

\twoCol {
    \subsection{Opérations sur les limites}\label{subsec:operations-sur-les-limites}

    \underline{Formes indéterminées :} $+\infty -\infty$, $0 \times \infty$, $\dfrac{\infty}{\infty}$, $\dfrac{0}{0}$, $\infty^0$, $0^0$, $1^{\infty}$.\\
    \underline{Polynômes et fractions rationnelles :} termes de plus haut degré.\\

    \subsection{Branches infinies}\label{subsec:branches-infinies}

    Si $\limm{+\infty} (f(x) - (ax+b)) = 0$ \\
    $y = ax + b$ est asymptote oblique à $C_f$ en $+\infty$.\\

    Si $\limm{+\infty} \dfrac{f(x)}{x} = a$ et $\limm{+\infty} (f(x) -ax) = \pm \infty$ :\\
    $f$ admet une branche parabolique de direction asymptotique à la droite $y = ax$ en $+\infty$.\\
    Si $a = \pm \infty$, la direction asymptotique est l'axe des ordonnées.\\
}{

    \subsection{Ordre et limites}\label{subsec:ordre-et-limites}
    Si $f$ croissante (resp. décroissante) sur $I =\ ]a, b[$\\
    où $a, b \in [-\infty, +\infty]$ :
    \begin{itemize}
        \item Si $f$ majorée sur $I$, elle admet une limite à gauche finie en $b$.
        \item Si $f$ non majorée sur $I$, $\limm{b^-} f(x) = +\infty$.
        \item Dans les deux cas $\limm{b^-} f(x) = \sup f(x)$.
    \end{itemize}
    \phantom{s}\\
    \underline{Croissance comparées :}
    \begin{itemize}
        \item Les puissance du logarithme sont négligeables devant les fonctions puissances positives.
        \item Les fonctions puissance sont négligeables devant les puissances positives de l'exponentielle.
    \end{itemize}
    \phantom{s}\\
    \underline{Ordre et limites :}
    Si $f \le g$, $\limm{x_0} f(x) \le \limm{x_0} g(x)$.\\
    On peut étendre ceci pour définir le théorème de l'encadrement.
}


\section{Continuité d'une fonction}\label{sec:continuite-d'une-fonction}

\subsection{Continuité en un point}\label{subsec:continuite-en-un-point}

$f$ continue en $x_0 \iff \limm{x_0} f(x) = f(x_0)$.
Sinon, $f$ peut être continue uniquement à gauche ou à droite de $x_0$.

\subsection{Prolongement par continuité}\label{subsec:prolongement-par-continuite}

Si $x_0 \not\in D_f$ mais que f admet une limite $l$ en $x_0$, on peut prolonger $f$ par continuité en définissant $\tilde{f}(x_0) = l$.

\subsection{Opérations}\label{subsec:operations}

Une combinaison linéaire, division ou composition de fonctions continues est continue.\\
Si $f$ est continue en $x_0$ et si $f(x_0) > 0$ alors $f(x) > 0$ au voisinage de $x_0$.


\section{Comparaison locale de deux fonctions}\label{sec:comparaison-locale-de-deux-fonctions}

\subsection{Négligeabilité}\label{subsec:negligeabilite}

$f$ négligeable devant $g$ : $f \underset{x_0}{=} o(g) \iff \limm{x_0} \dfrac{f}{g} = 0$\\
La négligeabilité se conserve par transitivité ou opérations d'addition, multiplication, division et puissance.

\subsection{Équivalence de fonctions}\label{subsec:equivalence-de-fonctions}

$f$ et $g$ équivalentes : $f \underset{x_0}{\sim} g \iff \limm{x_0} \dfrac{f}{g} = 1$\\
L'équivalence est symétrique, transitive et réflexive, et se conserve aussi par multiplication, division et puissance.\\

$\limm{x_0} f(x) = l \iff f \underset{x_0}{\sim} l$\\
$f \underset{x_0}{\sim} g \iff (f - g) \underset{x_0}{=} o(g) \underset{x_0}{=} o(f)$

\subsubsection{Équivalents usuels}

\threeCol{
    \begin{itemize}
        \item $e^x \underset{0}{\sim} 1 + x$
        \item $\ln(1+x) \underset{0}{\sim} x$
        \item $(1+x)^\alpha \underset{0}{\sim} 1 + \alpha x$
    \end{itemize}
}{
    \begin{itemize}
        \item $\cos x \underset{0}{\sim} 1 - \dfrac{x^2}{2}$
        \item $\sin x \underset{0}{\sim} x$
        \item $\tan x \underset{0}{\sim} x$
        \item $\arcsin x \underset{0}{\sim} x$
        \item $\arctan x \underset{0}{\sim} x$
    \end{itemize}
}{
    \begin{itemize}
        \item $\cosh x \underset{0}{\sim}  1 + \dfrac{x^2}{2}$
        \item $\sinh x \underset{0}{\sim} x$
        \item $\tanh x \underset{0}{\sim} x$
        \item $\cosh x \underset{+\infty}{\sim} \dfrac{e^x}{2}$
        \item $\sinh x \underset{+\infty}{\sim} \dfrac{e^x}{2}$
    \end{itemize}
}
