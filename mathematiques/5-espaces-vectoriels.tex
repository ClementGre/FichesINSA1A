\section{Structure d'espace vectoriel}\label{sec:structure-d'espace-vectoriel}
  
  Un ensemble $E$ est un $\mathbb K$-espace vectoriel ($\mathbb K$ désigne $\mathbb{R}$ ou $\mathbb{C}$) si il est muni d'une loi interne $\textbf{+}$ et d'une loi externe \textbf{$\cdot$}.\\
  
  On redéfinit les lois de bases (commutativité et associativité de l'addition, élément neutre\ldots).


\section{Sous espaces vectoriels}\label{sec:sous-espaces-vectoriels}
  
  $F$ est un sous-espace vectoriel de $E$ si $F \subset E$ avec $(E, +, \cdot)$ et $(F, +, \cdot)$ des $\mathbb K$-espace vectoriel.\\
  
  \textbf{Stabilité par combinaison linéaire :}
  \begin{equation}
    F \text{ est un s.e.v de } E \iff
    \begin{cases}
      F \neq \emptyset \text{ (Un s.e.v. contient toujours le vecteur nul)}\\
      \forall (\alpha, \beta, \vec u, \vec v) \in \mathbb{K}^2 \times F^2,\ \alpha \vec u + \beta \vec v \in F\\
      \text{(F est stable par combinaison linéaire)}
    \end{cases}\label{eq:equation2}
  \end{equation}
  
  L'intersection de deux s.e.v est un s.e.v.\ Ce n'est pas le cas pour l'union.\\
  
  \textbf{Sous-espace vectoriel engendré par A} ($Vect(A)$) : c'est le plus petit s.e.v. de $E$ contenant $A$.\\
  Par convention : $Vect(\emptyset) = {\vec{0_E}}$\\
  On dit que $Vect(A)$ est constitué de toutes les combinaisons linéaires des vecteurs de $A$.


\section{Dimension d'un espace vectoriel}\label{sec:dimension-d'un-espace-vectoriel}
  
  Une famille de vecteurs de $E$ est \underline{libre} s'ils sont linéairement indépendants (seul la combinaison linéaire avec des coefficients nuls mène à $\vec 0$).\\
  
  Sinon, la famille est \underline{liée}, les vecteurs sont alors linéairement dépendants (il existe des coefficients non tous nuls tel que la combinaison linéaire de ces vecteurs soit nulle).\\
  
  Deux vecteurs formant une famille liée sont colinéaires.
  Trois vecteurs formant une famille liée sont coplanaires.\\
  
  \begin{itemize}
    \item Une famille de vecteurs est liée $\iif$ l'un des vecteurs est combinaison linéaire des autres.
    \item Si une famille de vecteurs contient le vecteur nul, elle est liée.
    \item Une famille extraite d'une famille libre est libre.
    \item Une famille contenant une famille liée est liée.\\
  \end{itemize}
  
  Une famille de vecteurs de $E$ est \underline{génératrice} si tout vecteur de $E$ est combinaison linéaire de vecteurs de la famille.\\
  Une famille libre et génératrice de $E$ est une \underline{base} de $E$.\\
  
  
  