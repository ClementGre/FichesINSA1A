\section{Structure d'espace vectoriel}\label{sec:structure-d'espace-vectoriel}
  
  Un ensemble $E$ est un $\mathbb K$-espace vectoriel ($\mathbb K$ désigne $\mathbb{R}$ ou $\mathbb{C}$) si il est muni d'une loi interne $\textbf{+}$ et d'une loi externe \textbf{$\cdot$}, respectant les 8 axiomes de base :\\
  Commutativité et associativité de l'addition, élément neutre de la loi externe, existence d'un symétrique, multiplication par 1 (identité), ordre de la multiplication, distributivité à gauche et à droite.


\section{Sous espaces vectoriels}\label{sec:sous-espaces-vectoriels}
  
  $F$ est un sous-espace vectoriel de $E$ si $F \subset E$ avec $(E, +, \cdot)$ et $(F, +, \cdot)$ des $\mathbb K$-espace vectoriel.\\
  
  \textbf{Stabilité par combinaison linéaire :}
  \begin{equation}
    F \text{ est un s.e.v de } E \iff
    \begin{cases}
      F \neq \emptyset \text{ (Un s.e.v. contient toujours le vecteur nul)}\\
      \forall (\alpha, \beta, \vec u, \vec v) \in \mathbb{K}^2 \times F^2,\ \alpha \vec u + \beta \vec v \in F\\
      \text{(F est stable par combinaison linéaire)}
    \end{cases}\label{eq:equation2}
  \end{equation}
  
  L'intersection de deux s.e.v est un s.e.v.\ Ce n'est pas le cas pour l'union.\\
  
  \textbf{Sous-espace vectoriel engendré par A} ($\Vect(A)$) : c'est le plus petit s.e.v. de $E$ contenant $A$.\\
  Par convention : $\Vect(\emptyset) = {\vec{0_E}}$\\
  On dit que $\Vect(A)$ est constitué de toutes les combinaisons linéaires des vecteurs de $A$.


\section{Dimension d'un espace vectoriel}\label{sec:dimension-d'un-espace-vectoriel}
  
  \subsection{Familles libres, génératrices, bases}\label{subsec:familles-libres-generatrices-bases}
    
    Une famille de vecteurs de $E$ est \underline{libre} s'ils sont linéairement indépendants (seul la combinaison linéaire avec des coefficients tous nuls mène à $\vec 0$).\\
    
    Sinon, la famille est \underline{liée}, les vecteurs sont alors linéairement dépendants (il existe des coefficients non tous nuls tel que la combinaison linéaire de ces vecteurs soit nulle).\\
    
    Deux vecteurs formant une famille liée sont colinéaires.
    Trois vecteurs formant une famille liée sont coplanaires.\\
    
    \begin{itemize}
      \item Une famille de vecteurs est liée $\iff$ l'un des vecteurs est combinaison linéaire des autres.
      \item Si une famille de vecteurs contient le vecteur nul, elle est liée.
      \item Une famille extraite d'une famille libre est libre.
      \item Une famille contenant une famille liée est liée.\\
    \end{itemize}
    
    Une famille de vecteurs $(\vec{v_1}, \dots, \vec{v_p})$ d'un e.v. $E$ est \underline{génératrice} de $E$ si tout vecteur de $E$ est combinaison linéaire de vecteurs de la famille (Ce qui équivaut à dire que $E = \Vect(\vec{v_1}, \dots, \vec{v_p})$).\\
    Une famille libre et génératrice de $E$ est une \underline{base} de $E$ (Elle est canonique si chaque vecteur a une unique composante égale à 1).\\
  
  \subsection{Dimension finie}\label{subsec:dimension-finie}
    
    Un e.v. $E$ est de dimension finie s'il existe une famille finie de vecteurs génératrice de $E$.
    Sinon, $\dim(E) = +\infty$\\
    
    Soit $E$ un e.v. de dimension finie, on peut compléter une famille libre de $E$ par des vecteurs bien choisis d'une famille génératrice de $E$ pour obtenir une base de $E$.\\
    
    \newpage
    Soit $n = \dim(E)$ :
    \begin{itemize}
      \item Toute famille libre de E contient au plus $n$ vecteurs.
      \item Toute famille génératrice de E contient au moins $n$ vecteurs.
      \item Toute famille libre de $n$ vecteurs de $E$ est une base de $E$.
      \item Toute famille génératrice de $n$ vecteurs de $E$ est une base de $E$.
    \end{itemize}
    $\Rightarrow$ Soit $F$ une famille de $\dim(E)$ vecteurs : $F$ est une base de E $\iff$ $F$ est libre $\iff$ $F$ est génératrice de $E$\\
    
    \twoCol{
      \subsection{Sous-espace en dimension finie}\label{subsec:sous-espace-en-dimension-finie}
      
      Soit $F$ et $G$ des s.e.v de $E$ de dimension finie.
      \begin{itemize}
        \item $\dim E \ge \dim F$
        \item $\dim F = \dim E \iff F = E$
        \item $\dim(F \cap G) \le \min(\dim F, \dim G)$
        \item Si $F \subset G$ et $\dim F = \dim G$ alors $F = G$
      \end{itemize}
    }{
      
      \subsection{Rang d'une famille de vecteurs}\label{subsec:rang-d-une-famille-de-vecteurs}
      
      Soit $F$ une famille de $p$ vecteurs d'un e.v. $E$ :\\
      $\rg(F) = \dim(\Vect(F))$
      \begin{itemize}
        \item Si $F$ est libre, $\rg(F) = p$\quad Sinon, $\rg(F) < p$
        \item $\rg(F) = \dim E \iff$ $F$ est génératrice de $E$
        \item $\rg(F) = p = \dim E \iff$ $F$ est une base de $E$
      \end{itemize}
    }
  
  \subsection{Méthode des zéros échelonnés}\label{subsec:methode-des-zeros-echelonnes}
    
    Une famille est échelonnée relativement à une base si dans l'écriture des vecteurs exprimés dans la base, chaque vecteur commence par au moins un zéro de plus que le précédent.\\
    Toute famille de vecteurs tous non nuls échelonnée relativement à une base est libre.\\
    
    On peut échelonner des vecteurs en les écrivant en colonne (comme dans les systèmes), mais en faisant des combinaisons linéaires avec les colonnes (et non avec les lignes).
    On a donc pas d'équivalence, mais uniquement une égalité de rang $\rg(F)$.\\
    
  
    
    
  