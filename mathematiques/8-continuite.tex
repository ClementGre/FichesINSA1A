\section{Continuité et grand théorèmes}\label{sec:continuite-et-grand-theoremes}

\subsection{Fonctions continues sur un intervalle}\label{subsec:fonctions-continues-sur-un-intervalle}

$f$ est continue sur $I$ $\iff$ $f$ est continue en tout point de $I$.\\
On démontre souvent la continuité sur un intervalle par opérations de fonctions continues.\\

\textbf{Théorème des valeurs intermédiaires :}\\
Si $f$ est continue sur un intervalle $[a,\ b]$, alors pour tout réel $k$ compris entre $f(a)$ et $f(b)$, il existe un réel $c$ dans l'intervalle $[a,\ b]$ tel que $f(c) = k$.\\

\underline{Corollaire 1 :} Si $f(a)$ et $f(b)$ sont de signe opposés, alors il existe un réel $c$ dans l'intervalle $[a,\ b]$ tel que $f(c) = 0$.\\
\underline{Corollaire 2 :} L'image de $[a,\ b]$ est in intervalle $[f(a),\ f(b)]$.\\

\textbf{Définition d'un segment :} C'est un intervalle fermé borné, du type $[a,\ b]$\\

\textbf{Théorème des valeurs extrêmes :}\\
Une fonction continue sur un segment est bornée et atteint ses bornes.\\
\underline{Corollaire :} Si $f$ est continue, l'image d'un segment est un segment.\\

\subsection{Continuité et bijection}\label{subsec:continuite-et-bijection}

Si $f$ est continue et strictement monotone sur un intervalle $I$, alors :
\begin{itemize}
    \item $f$ est une bijection de $I$ sur $f(I)$.
    \item $f^{-1}$ est continue et strictement monotone sur $f(I)$, de même sens de variation que $f$.
\end{itemize}

\section{Application : Fonctions trigonométriques}\label{sec:application---fonctions-trigonometriques}

\subsection{La fonction arc sinus}\label{subsec:la-fonction-arc-sinus}

Restriction de sinus sur $[-\frac{\pi}{2},\ \frac{\pi}{2}]$ : strictement croissante, réalise une bijection sur $[-1,\ 1]$.\\
Bijection réciproque : arcsin, continue et croissante sur $[-1,\ 1]$.

\subsection{La fonction arc cosinus}\label{subsec:la-fonction-arc-cosunus}

Restriction de cosinus sur $[0,\ \pi]$ : strictement décroissante, réalise une bijection sur $[-1,\ 1]$.\\
Bijection réciproque : arccos, continue et strictement décroissante sur $[-1,\ 1]$.

\subsection{La fonction arc tangente}\label{subsec:la-fonction-arc-tangente}

Restriction de tangente sur $[-\frac{\pi}{2},\ \frac{\pi}{2}]$ : strictement croissante, réalise une bijection sur $\mathbb{R}$.\\
Bijection réciproque : arctan, continue et croissante sur $\mathbb{R}$.



