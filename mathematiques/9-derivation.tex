\section{Dérivées successives}\label{sec:deriv-es-successives}

    Fonction de classe $\mathcal{C}^n$ : $n$ fois dérivable et $f^{(n)}$ continue.\\
    Une somme, produit ou quotient de fonctions $\mathcal{C}^n$ reste $\mathcal{C}^n$ sur un intervalle $I$.\\
    $\exp$, $\ln$, $\sin$, $\cos$, polynômes et puissances sont $\mathcal{C}^{+\infty}$.


\section{Dérivabilité des fonctions réciproques}\label{sec:derivabilite-des-fonctions-reciproques}

    Soit $f$ continue strictement monotone. $f$ réalise donc une bijection.\\

    Si $f$ est dérivable en $a$ et $f'(a) \neq 0$, alors $f^{-1}$ est dérivable en $f(a) = b$.\\

    On a alors $(f^{-1})'(b) = \dfrac{1}{f'(a)} = \dfrac{1}{f' \circ f^{-1}(b)}$\\

    Cela permet de démontrer que $\arcsin'(x) = \dfrac{1}{\sqrt{1 - x^2}}$, $\arccos'(x) = -\dfrac{1}{\sqrt{1 - x^2}}$ et $\arctan'(x) = \dfrac{1}{1 + x^2}$.


\section{Fonctions à valeurs complexes}\label{sec:fonctions-a-valeurs-complexes}

    On peut décomposer la fonction en deux fonctions à valeurs réelles.
    \vspace{3pt}\\
    $f$ est dérivable si $\Re(f)$ et $\Im(f)$ le sont et $f'(x) = (\Re(f))' + i (\Im(f))'$.


\section{Théorème des accroissements finis (TAF)}\label{sec:theoreme-des-accroissements-finis-(taf)}

    \underline{Théorème de Rolle :}\\
    Soit $f$ continue sur $[a,\ b]$ et dérivable sur $]a,\ b[$ tel que $f(a) = f(b)$,\\
    alors il existe $c \in\ ]a,\ b[$ tel que $f'(c) = 0$.\\
    ($f$ admet au moins un extremum sur $]a,\ b[$)\\

    \underline{Théorème des accroissements finis :}\\
    Soit $f$ continue sur $[a,\ b]$ et dérivable sur $]a,\ b[$,\\
    alors il existe $c \in\ ]a,\ b[$ tel que $f(b) - f(a) = f'(c) (b - a)$.\\
    On note aussi $f'(c) = \dfrac{f(b) - f(a)}{b - a}$


\section{Applications du TAF}\label{sec:applications-du-taf}

    \subsection{Sens de variation d'une fonction et signe de la dérivée}\label{subsec:sens-de-variation-d'une-fonction-et-signe-de-la-derivee}
        Le signe de la dérivée donne la monotonie d'une fonction continue dérivable et vice versa.\\
        \underline{Stricte monotonie} de $f$ continue : $f'$ de signe constant ne s'annulant qu'en un nombre fini de points.

        \twoCol {
            \subsection{Extrema d'une fonction dérivable}\label{subsec:extrema-d'une-fonction-derivable}
            Une fonction possède un extremum local en $a$ s'il existe un voisinage $\mathcal{V}$ de $a$ où $\forall x \in \mathcal{V},\ f(x) \le f(a)$.\\
            (ou $\ge$ pour un minimum local).\\

            Dans ce cas, on a nécessairement $f'(a) = 0$.\\
            (Pour une fonction dérvable).\\

            Un réel $a$ pour lequel $f'(a) = 0$ est un point critique de $f$.\\
            Ça n'est un extremum que si $f'$ change de signe.
        }{
            \subsection{Théorème de la limite dérivée}\label{subsec:theoreme-de-la-limite-derivee}
            \begin{itemize}
                \item Si $\limm{x_0} f'(x) = l$, alors $f$ dérivable en $x_0$ et $f'(x_0) = l$.
                \vspace{5pt}
                \item Si $l = \pm \infty$, alors $f$ n'est pas dérivable en $x_0$\\
                (tangeante verticale).
                \vspace{5pt}
                \item Si $f'$ a une limite à gauche et à droite différente en $a$, alors $f$ n'est pas dérivable en $a$.
                \vspace{5pt}
                \item Si $f'$ n'admet pas de limite en $a$, on ne peut rien conclure.
            \end{itemize}

        }
