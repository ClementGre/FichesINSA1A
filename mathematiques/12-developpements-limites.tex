\section{Notion de développement limité}

Une fonction admet un développement limité d'ordre $n$ en $a$ s'il existe des réels $a_0, a_1, \ldots, a_n$ tels que :\\
\[f(x) \underset{x \rightarrow a}{=} \underbrace{a_0 + a_1 (x - a) + a_2 (x - a) + \ldots + a_n (x - a)}_\re{Partie régulière} + \underbrace{o((x - a)^n)}_\re{Reste}\]

\twoCol {
    Pour un DL en $0$ :
    \begin{itemize}
        \item Si $f$ est paire, alors $a_{2k+1} = 0$.
        \item Si $f$ est impaire, alors $a_{2k} = 0$.
    \end{itemize}
}{
    La partie régulière du $DL$ en 0 est unique.\\
    Le DL peut être tronqué à l'ordre $p \le n$.\\
    $f$ est équivalent en $a$ à son terme de plus petit degré.
}
\vspace{7pt}\\
\textbf{DL et régularité d'une fonction :}\\
$f$ admet un $\re{DL}_0$ en $a$ $\iff$ $f$ admet une limite finie en $a$ (continue ou prolongeable par continuité en $a$).\\
$f$ admet un $\re{DL}_1$ en $a$ $\iff$ $f$ est dérivable en $a$ et $f'(a) = a_1$.\\
Pou un DL d'ordre $n \ge 2$, $f$ n'est pas forcément dérivable $n$ fois en $a$.\\


\section{Formule de Taylor-Young}

Soit $f$ définie sur un voisinage de $a$ et dérivable $n$ fois, le $\re{DL}_n$ de $f$ en $a$ s'écrit :\\
\[f(x) \underset{x \rightarrow a}{=} \sum_{k=0}^n \frac{f^{(k)}(a)}{k!} (x - a)^k + o((x - a)^n)\]

On peut donc identifier les coefficients $a_k = \dfrac{f^{(k)}(a)}{k!}$. Ainsi, si $f$ est $C^\infty$ au voisinage de $a$, alors elle admet un DL à tout ordre en $a$. La réciproque est fausse.\\

\textbf{DL usuels :}

\begin{align*}
    e^x &\underset{x \rightarrow 0}{=} 1 + x + \dfrac{x^2}{2!} + \dfrac{x^3}{3!} + \dots + \dfrac{x^n}{n!} + o(x^n) \underset{x \rightarrow 0}{=} \sum_{k=0}^n \frac{x^k}{k!} + o(x^n)\\
    \ch(x) &\re{ : tous les ordres paires de l'exponentielle}\\
    \sh(x) & \re{ : tous les ordres impaires de l'exponentielle}\\
    \cos(x) &\underset{x \rightarrow 0}{=} 1 - \dfrac{x^2}{2!} + \dfrac{x^4}{4!} - \dots + (-1)^n \dfrac{x^{2n}}{(2n)!} + o(x^{2n+1}) \quad \re{(ch() avec alternance de signe)}\\
    \sin(x) &\underset{x \rightarrow 0}{=} x - \dfrac{x^3}{3!} + \dfrac{x^5}{5!} - \dots + (-1)^n \dfrac{x^{2n+1}}{(2n+1)!} + o(x^{2n+2}) \quad \re{(sh() avec alternance de signe)}\\
    \dfrac{1}{1 - x} &\underset{x \rightarrow 0}{=} 1 + x + x^2 + \dots + x^n + o(x^n)\\
    \ln(1 + x) &\underset{x \rightarrow 0}{=} x - \dfrac{x^2}{2} + \dfrac{x^3}{3} - \dots + (-1)^{n-1} \dfrac{x^n}{n} + o(x^n)\\
    (1+x)^\alpha &\underset{x \rightarrow 0}{=} 1 + \alpha x + \dfrac{\alpha(\alpha-1)}{2!} x^2 + \dots + \dfrac{\alpha(\alpha-1)\dots(\alpha-n+1)}{n!} x^n + o(x^n)\\
\end{align*}





