\section{Notion de développement limité}
	
	Une fonction admet un développement limité d'ordre $n$ en $a$ s'il existe des réels $a_0, a_1, \ldots, a_n$ tels que :\\
	\[f(x) \underset{x \rightarrow a}{=} \underbrace{a_0 + a_1 (x - a) + a_2 (x - a) + \ldots + a_n (x - a)}_\re{Partie régulière} + \underbrace{o((x - a)^n)}_\re{Reste}\]
	
	\twoCol {
		Pour un DL en $0$ :
		\begin{itemize}
			\item Si $f$ est paire, alors $a_{2k+1} = 0$.
			\item Si $f$ est impaire, alors $a_{2k} = 0$.
		\end{itemize}
	}{
		La partie régulière du $DL$ en 0 est unique.\\
		Le DL peut être tronqué à l'ordre $p \le n$.
	}
	
