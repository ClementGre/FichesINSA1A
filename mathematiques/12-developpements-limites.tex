\section{Notion de développement limité}

    Une fonction admet un développement limité d'ordre $n$ en $a$ s'il existe des réels $a_0, a_1, \ldots, a_n$ tels que :\\
    \[f(x) \underset{x \rightarrow a}{=} \underbrace{a_0 + a_1 (x - a) + a_2 (x - a) + \ldots + a_n (x - a)}_\re{Partie régulière} + \underbrace{o((x - a)^n)}_\re{Reste}\]

    \twoCol {
        Pour un DL en $0$ :
        \begin{itemize}
            \item Si $f$ est paire, alors $a_{2k+1} = 0$.
            \item Si $f$ est impaire, alors $a_{2k} = 0$.
        \end{itemize}
    }{
        La partie régulière du $DL$ en 0 est unique.\\
        Le DL peut être tronqué à l'ordre $p \le n$.\\
        $f$ est équivalent en $a$ à son terme de plus petit degré.
    }
    \vspace{7pt}\\
    \textbf{DL et régularité d'une fonction :}\\
    $f$ admet un $\re{DL}_0$ en $a$ $\iff$ $f$ admet une limite finie en $a$ (continue ou prolongeable par continuité en $a$).\\
    $f$ admet un $\re{DL}_1$ en $a$ $\iff$ $f$ est dérivable en $a$ et $f'(a) = a_1$.\\
    Pou un DL d'ordre $n \ge 2$, $f$ n'est pas forcément dérivable $n$ fois en $a$.\\


\section{Formule de Taylor-Young}

    Soit $f$ définie sur un voisinage de $a$ et dérivable $n$ fois, le $\re{DL}_n$ de $f$ en $a$ s'écrit :\\
    \[f(x) \underset{x \rightarrow a}{=} \sum_{k=0}^n \frac{f^{(k)}(a)}{k!} (x - a)^k + o((x - a)^n)\]

    On peut donc identifier les coefficients $a_k = \dfrac{f^{(k)}(a)}{k!}$. Ainsi, si $f$ est $C^\infty$ au voisinage de $a$, alors elle admet un DL à tout ordre en $a$. La réciproque est fausse.\\

    \textbf{DL usuels :}
    \begin{align*}
        e^x &\underset{x \rightarrow 0}{=} 1 + x + \dfrac{x^2}{2!} + \dfrac{x^3}{3!} + \dots + \dfrac{x^n}{n!} + o(x^n) \underset{x \rightarrow 0}{=} \sum_{k=0}^n \frac{x^k}{k!} + o(x^n)\\
        \ch(x) &\re{ : tous les ordres paires de l'exponentielle}\\
        \sh(x) & \re{ : tous les ordres impaires de l'exponentielle}\\
        \cos(x) &\underset{x \rightarrow 0}{=} 1 - \dfrac{x^2}{2!} + \dfrac{x^4}{4!} - \dots + (-1)^n \dfrac{x^{2n}}{(2n)!} + o(x^{2n+1}) \quad \re{(ch() avec alternance de signe)}\\
        \sin(x) &\underset{x \rightarrow 0}{=} x - \dfrac{x^3}{3!} + \dfrac{x^5}{5!} - \dots + (-1)^n \dfrac{x^{2n+1}}{(2n+1)!} + o(x^{2n+2}) \quad \re{(sh() avec alternance de signe)}\\
        \dfrac{1}{1 - x} &\underset{x \rightarrow 0}{=} 1 + x + x^2 + \dots + x^n + o(x^n)\\
        \ln(1 + x) &\underset{x \rightarrow 0}{=} x - \dfrac{x^2}{2} + \dfrac{x^3}{3} - \dots + (-1)^{n-1} \dfrac{x^n}{n} + o(x^n)\\
        (1+x)^\alpha &\underset{x \rightarrow 0}{=} 1 + \alpha x + \dfrac{\alpha(\alpha-1)}{2!} x^2 + \dots + \dfrac{\alpha(\alpha-1)\dots(\alpha-n+1)}{n!} x^n + o(x^n)
    \end{align*}

    \newpage


\section{Opérations sur les développements limités}

    \twoCol {
        \subsection{Somme et produit}

        On peut faire la somme et le produit de deux DL$_n$.
        \smallskip

        Pour le produit, on tronque le DL à l'ordre $n$.
    }{
        \subsection{Quotient}

        Si le dénominateur a un coefficient constant non nul, on peut faire le quotient de deux DL$_n$ avec l'algorithme de division suivant les puissances croissantes.
    }

    \subsection{Changements de variable et composition de DL}

        On peut faire un changement de variable polynomial dans un DL si les deux variables tendent vers $0$.
        \medskip

        Autrement, on peut composer deux DL$_n$ en $0$ si le premier tend bien vers $0$ (coefficient constant nul).

        \twoCol {
            \subsection{Intégration de DL}

            On peut intégrer un DL$_n$ en $0$ en ajoutant une constante $F(0)$.
            \smallskip

            Toute primitive de $f$ admettra le DL intégré.
        }{
            \subsection{Dérivation de DL}

            On peut dériver un DL$_n$ en $0$ si on sait que $f'$ admet un DL$_{n-1}$ en $0$.
            \smallskip

            Cela ne prouve pas que $f'$ admet un DL$_n$ en $0$.
        }


\section{Étude locale d'une courbe}

    \subsection{Étude locale au voisinage de $x = a$}

        L'équation de la tangente en $a$ se trouve à partir du DL$_1$ de $f$ en $a$.\\

        La position de $f$ par rapport à sa tangente au voisinage de $a$ dépend du coefficient qui suit\\(pas forcément celui du degré $2$) :\\
        \begin{itemize}
            \item Si l'ordre $n$ est pair, alors le signe de $a_n (x - a)^n$ sera celui de $a_n$.\\
            Si $a_n > 0$, alors $f$ est au dessus de sa tangente, sinon elle est en dessous.
            \item Si $n$ est impair, $f$ traversera sa tangente en $a$ (on étudie le signe de $a_n (x - a)^n$).
        \end{itemize}

    \subsection{Développement limité généralisé (DLG) en $\pm \infty$ et branche infinie}

        $f$ admet un $DL$ à l'ordre $n$ en $\pm \infty$ s'il existe des réels $a_0, \dots, a_n$ tels que :\\
        \[f(x) \underset{\pm \infty}{=} a_0 + \frac{a_1}{x} + \dots + \frac{a_n}{x^n} + o\left(\frac{1}{x^n}\right)\]

        Pour calculer des DLG, en $\pm \infty$, on fait souvent le changement de variable $x = \dfrac{1}{t}$ et on calcule un DL en $0$.

        Si $f(x) \underset{\pm \infty}{=} ax + b + o(1)$, alors un DL de $\dfrac{f(x)}{x}$ en $\pm \infty$ est $a + \dfrac{b}{x} + o\left(\dfrac{1}{x}\right)$.\\

        Ainsi, la courbe de $f$ admet comme asymptote la droite $y = ax + b$ en $\pm \infty$.
        Le signe du terme suivant permet de déterminer la position locale de $f$ par rapport à son asymptote.
