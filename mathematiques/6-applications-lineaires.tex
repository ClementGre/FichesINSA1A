\section{L'espace vectoriel $\mathcal{L}(E, F)$}\label{sec:l'espace-vectoriel-$mathcal{l}(e-f)$}
  
  $f$ est une application linéaire si $f \in \mathcal{L}(E, F)$\\
  $\iff \forall (\alpha, \beta) \in \mathbb{K}^2, \forall (\vec u, \vec v) \in E^2, f(\alpha \cdot_E \vec u +_E  \beta \cdot_E \vec v) = \alpha \cdot_F f(\vec u) +_F \beta \cdot_F f(\vec v)$

  Toute réciproque d'une bijection linéaire, combinaison linéaire d'une application linéaire ou composée de deux applications linéaires \textbf{est linéaire}.\\
  Une application linéaire bijective de $E$ dans $F$ est un isomorphisme de $E$ sur $F$.
  $E$ et $F$ sont isomorphes.

\section{Image par une application linéaire}\label{sec:image-par-une-application-lineaire}


\section{Applications linéaires en dimension finie}\label{sec:applications-lineaires-en-dimension-finie}
