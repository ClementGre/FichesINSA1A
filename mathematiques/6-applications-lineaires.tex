\section{L'espace vectoriel $\mathcal{L}(E, F)$}\label{sec:l'espace-vectoriel-$mathcal{l}(e-f)$}
  
  On considérera $f$ une application linéaire tel que $f \in \mathcal{L}(E, F)$\\
  On a : $\forall (\alpha, \beta) \in \mathbb{K}^2, \forall (\vec u, \vec v) \in E^2, f(\alpha \cdot_E \vec u +_E  \beta \cdot_E \vec v) = \alpha \cdot_F f(\vec u) +_F \beta \cdot_F f(\vec v)$\\
  
  Toute réciproque d'une bijection linéaire, combinaison linéaire de deux applications linéaires ou composée de deux applications linéaires \textbf{est linéaire}.\\
  Une application linéaire bijective de $E$ dans $F$ est un isomorphisme de $E$ sur $F$.
  $E$ et $F$ sont isomorphes.\\
  Si $E = F$, c'est un endomorphisme bijectif ou automorphisme de $E$.


\section{Image par une application linéaire}\label{sec:image-par-une-application-lineaire}
  
  \twoCol{
    
    \subsection{Image et image réciproque}\label{subsec:image-et-image-reciproque}
    
    \begin{itemize}
      \item Image directe : image d'un ensemble.
      \item Image réciproque : antécédents d'un ensemble : $f^{-1}(A)$ (toujours définie contrairement à la réciproque).
    \end{itemize}
    \\
    L'image s'un s.e.v. est un s.e.v. dans l'ensemble de définition.
  }{
    
    \subsection{Noyau et image}\label{subsec:noyau-et-image}
    
    \begin{itemize}
      \item L'image de $f$ est le s.e.v $\Im(f) = f(E)$\\(Ensemble d'arrivée de $f$).
      \item Le noyau de $f$ est le s.e.v. $\Ker(f) = f^{-1}(\vec{0_F})$\\(Tous les $\vec u$ de $E$ ayant pour image $\vec 0$).
    \end{itemize}
    
    $f$ est surjective $\iff \Im(f) = F$.\\
    $f$ est injective $\iff \Ker(f) = \{\vec 0\}$.
  }
  
  \subsection{Image d'une famille de vecteurs}\label{subsec:image-d'une-famille-de-vecteurs}
    
    Soit $f \in \mathcal{L}(E, F)$ et $(\vec{u_1}, \dots, \vec{u_p})$ une famille de $E$ :
    \begin{itemize}
      \item Si $(\vec{u_1}, \dots, \vec{u_p})$ est liée, alors $(f(\vec{u_1}), \dots, f(\vec{u_p}))$ est liée.
      \item Si $(\vec{u_1}, \dots, \vec{u_p})$ est libre et $f$ injective, alors $(f(\vec{u_1}), \dots, f(\vec{u_p}))$ est libre.
      \item Si $(\vec{u_1}, \dots, \vec{u_p})$ est génératrice et $f$ surjective, alors $(f(\vec{u_1}), \dots, f(\vec{u_p}))$ est génératrice.
      \item Si $G = \Vect(\vec{u_1}, \dots, \vec{u_p})$, alors $f(G) = \Vect(f(\vec{u_1}), \dots, f(\vec{fu_p}))$.
    \end{itemize}


\section{Applications linéaires en dimension finie}\label{sec:applications-lineaires-en-dimension-finie}
  
  \subsection{Image d'une famille de vecteurs}\label{subsec:image-d'une-famille-de-vecteurs2}
    
    Soit $f \in \mathcal{L}(E, F)$ et $(\vec{e_1}, \dots, \vec{e_p})$ une base de $E$, de dimension finie :
    \begin{itemize}
      \item $(f(\vec{e_1}, \dots, f(\vec{e_p}))$ est libre dans $F$ ssi $f$ est injective.
      \item $(f(\vec{e_1}, \dots, f(\vec{e_p}))$ est génératrice dans $F$ ssi $f$ est surjective.
      \item $(f(\vec{e_1}, \dots, f(\vec{e_p}))$ est une base de $F$ ssi $f$ est un isomorphisme.
      \item Si $f$ injective, alors $\dim{F} \ge \dim{E}$
      \item Si $f$ surjective, alors $\dim{F} \le \dim{E}$
      \item Si $f$ est un isomorphisme, alors $\dim{F} = \dim{E}$
      \item Si $\forall i \in \llbracket 0~;~n \rrbracket$, $f(\vec{e_i}) = 0$, alors $f$ est l'application nulle.
      \item Si $(f, g) \in \mathcal{L}(E, F)$ et que $f(\vec{e_i}) = g(\vec{e_i})$, alors $f = g$.
      \item Soit $(\vec{\epsilon_1}, \dots, \vec{\epsilon_n})$ des vecteurs de $F$, alors il existe une unique A.L. $f$ telle que $\forall i \in \llbracket 0~;~n \rrbracket$, $f(\vec{e_i}) = \vec{\epsilon_i}$.
    \end{itemize}
    \ \\
    \twoCol{
      \subsection{Représentation analytique}\label{itm:representation-analytique}
      
      On peut écrire un système appelé représentation analytique de $f$, qui exprime chacun des coordonnés de $f(\vec{u})$ en fonction de combinaison linéaire des composantes de $\vec{u}$ et des coefficients de $f$.
    }{
      \subsection{Matrice d'une application linéaire}\label{itm:matrice-d'une-application-lineaire}
      
      On peut aussi écrire matriciellement :\\
      $\vec v = f(\vec u) \iff Y = AX$.\\
      Avec $A$ la matrice des coefficients de $f$ et $X$ la matrice colonne des composantes de $\vec u$.
    }
  
  \subsection{Rang d'une application linéaire}\label{subsec:rang-d'une-application-lineaire}
