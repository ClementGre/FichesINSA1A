\section{L'espace vectoriel $\mathcal{L}(E, F)$}\label{sec:l'espace-vectoriel-$mathcal{l}(e-f)$}
  
  $f$ est une application linéaire si $f \in \mathcal{L}(E, F)$\\
  $\iff \forall (\alpha, \beta) \in \mathbb{K}^2, \forall (\vec u, \vec v) \in E^2, f(\alpha \cdot_E \vec u +_E  \beta \cdot_E \vec v) = \alpha \cdot_F f(\vec u) +_F \beta \cdot_F f(\vec v)$\\
  
  Toute réciproque d'une bijection linéaire, combinaison linéaire de deux applications linéaires ou composée de deux applications linéaires \textbf{est linéaire}.\\
  Une application linéaire bijective de $E$ dans $F$ est un isomorphisme de $E$ sur $F$.
  $E$ et $F$ sont isomorphes.\\
  Si $E = F$, c'est un endomorphisme bijectif ou automorphisme de $E$.


\section{Image par une application linéaire}\label{sec:image-par-une-application-lineaire}
  
  \begin{itemize}
    \item Image directe : image d'un ensemble.
    \item Image réciproque : antécédents d'un ensemble : $f^{-1}(A)$ (toujours définis contrairement à la réciproque).
  \end{itemize}
  
  L'image s'un s.e.v. est un s.e.v. dans l'ensemble de définition.\\
  
  Soit une application linéaire $f \in \mathcal{L}(E, F)$ :
  \begin{itemize}
    \item L'image de $f$ est le s.e.v $\Im(f) = f(E)$ (Ensemble d'arrivée de $f$).
    \item Le noyau de $f$ est le s.e.v. $\Ker(f) = f^{-1}(\vec{0_F})$ (Tous les $\vec u$ de $E$ ayant pour image $\vec 0$).
  \end{itemize}
  
  $f$ est surjective $\iff \Im(f) = F$.\\
  $f$ est injective $\iff \Ker(f) = \{\vec 0\}$.


\section{Applications linéaires en dimension finie}\label{sec:applications-lineaires-en-dimension-finie}
