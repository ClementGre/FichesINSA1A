\twoCol[45] {
  
  
  \section{Définitions}\label{sec:definitions}
  
  $\deg(0) = -\infty$\\
  
  \textbf{Opérations :}
  \begin{itemize}
    \item $P = Q \Leftrightarrow$ même degré et mêmes coefs.
    \item $\deg(P + Q) \le \max(\deg(p), \deg(Q))$
    \item $\deg(P \times Q) = \deg(p) + \deg(Q)$
    \item $\deg(P \circ Q) \le \deg(p) \times \deg(Q)$
  \end{itemize}
}{
  \section{Division euclidienne et racines}\label{sec:division-euclidienne}
  
  Les polynômes peuvent être divisés par un polynôme non nul, au même titre que les réels.\\
  Soit $\alpha$ une racine de $P$, alors $P$ est divisible par $(X - \alpha)$.\\
  Un polynôme $P$ admet $n$ racines avec $n \le \deg(P)$.\\
  
  Un polynôme a la même limite en $+\infty$ et en $-\infty$ que son terme de plus haut degré.\\
  Si un polynôme $P$ est de degré impair, alors $P$ a au moins une racine réelle.
}


\section{Formule de Taylor}\label{sec:formule-de-taylor}
  
  La dérivée d'un polynôme $\displaystyle P = \sum^n_{k=0} a_k X^k$ est donnée par : $\displaystyle P' = \sum^n_{k=1} ka_k X^{k-1}$\\
  
  Pour la dérivée $k-$ième, on retrouve $k!$ pour le premier coefficient.
  Ainsi, on a $\forall\ k \le \deg(P)$, $P^{(k)}(0) = k!\ a_k$
  
  Ainsi, on a $\displaystyle P = \sum^n_{k=0} \dfrac{P^{(k)}(0)}{k!} X^k$\\
  
  Généralisation (formule de Taylor) : $\displaystyle P = \sum^n_{k=0} \dfrac{P^{(k)}(\alpha)}{k!} (X-\alpha)^k$


\section{Racines multiples}\label{sec:racines-multiples}
  
  $\alpha$ est une racine d'ordre de multiplicité $m$ de $P$ si $P = (X - \alpha)^m\ Q$ avec $Q(\alpha) \neq 0$.\\
  (Racine simple, double, triple, ...).\\
  
  La formule de Taylor permet de donner une caractérisation de la multiplicité :\\
  $\alpha$ est une racine de $P$ d'ordre de multiplicité $m$ ssi :
  \begin{equation}
    \label{eq:equation}
    \begin{cases}
      P(\alpha) = P'(\alpha) = \cdots = P^{(m-1)}(\alpha) = 0\\
      P^{(m)}(\alpha) \neq 0
    \end{cases}
  \end{equation}
  
  (Par exemple, pour une racine double, la tangente est horizontale, et pour une racine triple, il y a en plus un point d'inflexion).


\section{Factorisation}\label{sec:factorisation}
  
  Tout polynôme $P$ à coefficient dans $\mathbb{C}$ admet exactement $\deg(P)$ racines complexes, comptées avec leur ordre de multiplicité.\\
  
  Si $z$ est une racine complexe (non réelle) de $P$ et de multiplicité $m$, alors $\overline{z}$ est de même une racine de $P$ de multiplicité $m$.\\