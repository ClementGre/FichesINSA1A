\section{Somme de sous-espace vectoriels}\label{sec:somme-de-sous-espace-vectoriels}

    \subsection{Somme et somme directe}\label{subsec:somme-et-somme-directe}

        La somme de deux sev $F$ et $G$ est un sev $F + G = \{u_F + u_G,\ u_F \in F,\ u_G \in G\}$\\

        On a : $F + G = \Vect(F \cup G)$ ($F + G$ est le plus petit sev engendré par les sev $F$ et $G$).\\
        $F + G$ est directe si $F \cap G = \{\vec 0 \}$ On note $F \oplus G$.\\
        Tout élément de $F \oplus G$ s'écrit de façon unique comme somme d'un élément de $F$ et de $G$.

    \subsection{Supplémentaires}\label{subsec:supplementaires}
        $F$ et $G$ sont supplémentaires dans $E$ $\iff$ $E = F \oplus G$ (donc si  $E = F + G$ et $F \cap G = \{0_E\}$)\\

        Un sev peut avoir une infinité de supplémentaires.

    \subsection{En dimension finie}\label{subsec:en-dimension-finie}

        \underline{Formule de Grassman :}\\
        Soit $E$ un ev de dimension finie et $F$ et $G$ deux sev de $E$ :
        \[\dim(F + G) = \dim(F) + \dim(G) - \dim(F \cap G)\]

        $\mathcal{B}_F \cap \mathcal{B}_G$ est une base de $E$ $\iff$ $F \oplus G = E$.\\
        Dans un ev de dimension finie $n$, tout sev $F$ admet des supplémentaires de dimension $n - \dim(F)$.


\section{Projections vectorielles}\label{sec:projections-vectorielles}

    \[E = F \oplus G \iff \forall \vec x \in E,\ \exists ! \vec x_G \in G,\ \exists ! \vec x_F \in F,\ \vec x = \vec x_F + \vec x_G\]

    Ainsi, on appelle projection vectorielle sur $F$ parallèlement à $G$ l'application $p : E \to F$ définie par $p(\vec x) = \vec x_F$.\\

    Ainsi, si $E = F \oplus G$,
    \begin{itemize}
        \item $p$ est linéaire (endomorphisme de $E$)
        \item $G = \Ker(p)$ et $F = \Im(p)$
        \item $F = \left\{\vec x \in E,\ p(\vec x) = \vec x\right\} = \Ker(p - id_E)$ ($F$ est l'espace des invariants de $p$)
    \end{itemize}
    \vspace{10px}
    En dimension finie, si un endomorphisme $p$ de $E$ est une projection, alors $\Ker(p) \oplus \Im(p) = E$.\\

    Si $p$ est une projection vectorielle, alors $p \circ p = p$.\\
    Réciproquement, si $p$ est un endomorphisme de $E$ tel que $p \circ p = p$, alors $p$ est une projection vectorielle sur $\Im(p)$ parallèlement à $\Ker(p)$.\\
