\documentclass[13pt, twoside, a4paper, french]{report}

\usepackage{ficheslib}
\newcommand*{\getSubject}{Démarche scientifique et concepts fondamentaux}

\begin{document}
\title{\getSubject}
\author{Clément GRENNERAT}
\date{Septembre 2022}


\chapter{Grandeurs physiques, dimensions et unités}\label{ch:grandeur-physique-dimension-unite}
  
  
  \section{Grandeurs physiques}\label{sec:grandeur-physique}
    
    Grandeur = propriété d'un corps ou d'un phénomène (distance, durée, charge, force\ldots)
    Pour mesurer une grandeur, on établit le rapport entre cette grandeur et son unité de mesure.\\
    On marque donc $g = \dfrac{G}{u}$ ou $G = g . u$ avec G la grandeur et u l'unité.
    
    \subsection{Grandeurs vectorielles}\label{subsec:grandeurs-vectorielles}
      
      \textbf{4 paramètres :} point d'application et (direction, norme, sens) ou (x, y, z).
    
    \subsection{Grandeurs scalaires}\label{subsec:grandeurs-scalaires}
      
      Représenté par un nombre unique.
      
      \begin{tabular}[t]{|c|c|c|}
        \hline
        \headRow Non mesurable & \multicolumn{2}{c|}{Mesurable} \\\hline
        \multirow{2}{*}{Classement uniquement} & \headRow Extensive                       & Intensive           \\\cline{2-3}
        & Proportionnelle à la quantité de matière & Non proportionnelle \\\hline
      \end{tabular}%
  
  
  \section{Dimensions}\label{sec:dimensions}
    
    \begin{minipage}[t]{0.5\textwidth}%
      Dimension d'une grandeur : $\dim(G)$
      
      Il y a sept dimensions fondamentales.
      L'équation aux dimensions exprime la dimension d'une grandeur à partir des sept dimensions fondamentales\\
      
      Les angles n'ont pas de dimensions, ainsi que les arguments des fonctions $\sin$, $\cos$, $\tan$, $\ln$ et $\exp$.\\
      
      Sans dimension signifie que $\dim(G) = 1$
    \end{minipage}\hspace{0.03\textwidth}
    \begin{minipage}[t]{0.47\textwidth}%
      \begin{tabular}[t]{|c|c|}
        \hline
        \headRow Dimension fondamentale & Symbole  \\\hline
        Longueur                        & $L$      \\\hline
        Masse                           & $M$      \\\hline
        Temps                           & $T$      \\\hline
        Intensité électrique            & $I$      \\\hline
        Température                     & $\theta$ \\\hline
        Quantité de matière             & $N$      \\\hline
        Intensité lumineuse             & $J$      \\\hline
      \end{tabular}%
    \end{minipage}
  
  
  \section{Unités}\label{sec:unites}
    
    \begin{minipage}[t]{0.25\textwidth}%
      L’unité de mesure est une grandeur scalaire, définie par
      convention.\\
      
      On fait alors le rapport avec cette unité : $g = \dfrac{G}{u}$
    \end{minipage}\hspace{0.03\textwidth}
    \begin{minipage}[t]{0.62\textwidth}%
      \begin{tabular}[t]{|c|c|c|c|}
        \hline
        \headRow Dimension fondamentale & Symbole Dim & Unité      & Symbole unité \\\hline
        Longueur                        & $L$         & mètre      & $m$           \\\hline
        Masse                           & $M$         & kilogramme & $kg$          \\\hline
        Temps                           & $T$         & seconde    & $s$           \\\hline
        Intensité électrique            & $I$         & ampère     & $A$           \\\hline
        Température                     & $\theta$    & kelvin     & $K$           \\\hline
        Quantité de matière             & $N$         & mole       & $mol$         \\\hline
        Intensité lumineuse             & $J$         & candela    & $cd$          \\\hline
      \end{tabular}%
    \end{minipage}


\chapter{Incertitudes}\label{ch:incertitudes}
  
  
  \section{Définition erreur et incertitude}\label{sec:definition-erreur-et-incertitude}
    
    \subsection{Qu'est-ce qu'une erreur}\label{subsec:qu-est-ce-qu-une-erreur}
      
      \begin{minipage}[t]{0.50\textwidth}%
        \textbf{Erreur absolue :} $\delta g = m - g$ où $m$ est la mesure et $g$ la valeur exacte.
        
        \textbf{Erreur relative :} $\delta g_r = \dfrac{m - g}{g}$ où $m$ est la mesure et $g$ la valeur exacte.
      \end{minipage}\hspace{0.03\textwidth}
      \begin{minipage}[t]{0.47\textwidth}%
        L'erreur n'est pas connue (sinon, on aurait la valeur exacte).\\
        On s'intéresse donc à l'incertitude, qui a pour but d'estimer l'erreur de manière raisonnable.
      \end{minipage}
    
    \subsection{Origine des erreurs}\label{subsec:origine-des-erreurs}
      
      \begin{tabular}[t]{|l|l|l|}
        \hline
        \headRow Type d'erreur & Description                             & Exemple sur la mesure du volume d'un poly         \\\hline
        Matière                & Grandeur mal définie ou fluctuante      & Coins arrondis                                    \\\hline
        Méthode & \makecell[l]{Perturbation du système par \\ l’introduction d’un appareil de mesure}                 & Pied à coulisse qui écrase le poly             \\\hline
        Moyens           & Imperfections de l’appareil                         & Règle imparfaite \\\hline
        Main d’œuvre                 & Expérimentateur & Mauvaise lecture des graduations, parallaxe\ldots             \\\hline
        Milieu                 & Influence des conditions expérimentales & Taille dépend de la température\ldots             \\\hline
      \end{tabular}%
    
    \subsection{Les deux sortes d'erreurs}\label{subsec:les-deux-sortes-d'erreurs}
      
      \begin{minipage}[t]{0.50\textwidth}%
        \textbf{Erreur systématique :} erreur qui se répète identiquement à chaque mesure.
        Peut être due à la méthode, à la main-d'oeuvre ou aux moyens.\\
        
        \textbf{Erreur aléatoire :} erreur qui varie aléatoirement d'une mesure à l'autre.
        Peut être due aux 5 causes possibles.
      \end{minipage}\hspace{0.03\textwidth}
      \begin{minipage}[t]{0.47\textwidth}%
        \textbf{Mesures justes :} moyenne des mesures proche de la valeur vraie.\\
        \textbf{Mesures fidèles :} valeurs proches lors de mesures répétées.\\
        \textbf{Résolution d’un appareil :} plus petite variation décelable.
      \end{minipage}
    
    \subsection{Incertitude}\label{subsec:incertitude}
      
      Permet d’estimer la dispersion des résultats de mesure.\\
      
      \textbf{Incertitude absolue $\Delta g$ :} limite \textbf{supérieure raisonnable} estimée de la valeur absolue de l’erreur $\lvert \Delta g \rvert$ sur la mesure.\\ La valeur vraie appartient donc à $[g - \Delta g\ ;\ g + \Delta g]$
  
  
  \section{Estimation des incertitudes}\label{sec:estimation-des-incertitudes}
    
    \begin{minipage}[t]{0.65\textwidth}%
      \subsection{Mesure directe}\label{subsec:mesure-directe}
        
        Essayer de changer d'instrument de mesure, de méthode, ou de mesurer une grandeur étalon.
        
        \textbf{Incertitude de type A :} Série de mesure → étude statistique. (Valeurs souvent réparties selon une loi normale).\\
        Espérance (estimée par la limite de la moyenne) : $\displaystyle \mu = \lim_{n \to +\infty} = \dfrac{1}{n} \sum_{i=1}^{n} x_i$\\
        Écart-type expérimental : $\displaystyle \sigma = \sqrt{\dfrac{1}{n-1} \sum_{i=1}^{n} (g_i - \overline{g})^2}$\\
        Incertitude type (estimation de l'erreur) : $\Delta g = \dfrac{\sigma}{\sqrt{n}}$\\
        
        \textbf{Incertitude de type B :} Mesure unique (ou faible nombre de répétitions)\\
        → Estimation de l'erreur (diagramme des 5 M) et estimation des contributions.\\
        On obtiens l'incertitude maximale (somme).
    
    \end{minipage}\hspace{0.03\textwidth}
    \begin{minipage}[t]{0.32\textwidth}%
      \subsection{Mesure indirecte}\label{subsec:mesure-indirecte}
        
        \textbf{Méthode par encadrement :} on applique d'une part les incertitudes de manière à minimiser le résultat, puis de manière à le maximiser :
        \begin{gather*}
          g_{\min} < g < g_{\max}\\
          \min(f(x, y, z)) < g < \max(f(x, y, z))\\
          \Delta g = \frac{g_{\max} - g_{\min}}{2}\\
        \end{gather*}
        
        \textbf{Méthode par différentielle :}\\ Voir OMNI.
    
    \end{minipage}


\end{document}