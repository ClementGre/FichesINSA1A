\documentclass[13pt, twoside, a4paper, french]{report}

\usepackage{ficheslib}
\newcommand*{\getSubject}{Mécanique}

\begin{document}
    \title{\getSubject}
    \author{Clément GRENNERAT}
    \date{Septembre 2022}
    \pagestyle{non-chapter-style}


    \chapter{Statique du solide}\label{ch:statique-du-solide}


        \section{Définitions indispensables}\label{sec:definitions-indispensables}

            \subsection{Fluides et solides}\label{subsec:fluides-et-solides}

                \begin{outline}
                    \1 \textbf{Solide} : possède une forme et un volume propre, indéformable, la distance entre deux points quelconques reste constante.
                    \1 \textbf{Fluide :} n'a pas de forme propre
                    \2 \textbf{Liquide :} Volume propre
                    \2 \textbf{Gaz :} Pas de volume propre, occupe toute la place disponible
                \end{outline}

                \textbf{Point matériel ou masse ponctuelle :} point d'un volume nul et de masse non nulle.

            \subsection{Système et référentiel}\label{subsec:systeme-et-referentiel}

                Penser à définir le \textbf{système} et le \textbf{référentiel} au début du problème.

                Barycentre d'un système :
                \[\sum_{i=1}^n m_i\ \overrightarrow{GA_i} = \overrightarrow{0} \qquad \text{ou} \qquad \overrightarrow{OG} = \dfrac{\sum_{i=1}^n m_i\ \overrightarrow{OA_i}}{M_{total}}\]
                Pour un solide de masse $M$ :
                \[\overrightarrow{OG} = \dfrac{1}{M} * \iiint_{\text{solide}} {\rho\ \re d V\ \overrightarrow{OM}} \]

            \subsection{Modéliser des interactions par des forces}\label{subsec:modeliser-des-interactions-par-des-forces}

                Force : concept physique modélisant l'interaction entre deux systèmes (créant un mouvement, une déformation).\\

                \textbf{Troisième loi de Newton} (actions réciproques) : $\overrightarrow{F_{1\rightarrow2}} = - \overrightarrow{F_{2\rightarrow1}}$ (dans tout référentiel).\\

                \textbf{Deux types de forces :} force à distance ou de contact.
                \begin{outline}
                    \1 \underline{Interaction gravitationnelle} : $\overrightarrow{F_{g1\rightarrow2}} = - G\dfrac{m_1 m_2}{d^2}\overrightarrow{u_{1\rightarrow2}}$
                    \1 \underline{Interaction Électromagnétique} : $\overrightarrow{F_{e1\rightarrow2}} = \dfrac{q_1 q_2}{4\pi\epsilon_0 d^2}\overrightarrow{u_{1\rightarrow2}}$.
                    \1 \underline{Tension d'un fil} : toujours inconnue sauf si le fil est détendu.
                    \1 \underline{Rappel élastique d'un ressort} : $\vec{F} = - k (l - l_0)\vec{\imath}$.\\Avec $l_0$ la longueur à vide, $l$ la longueur étiré/comprimé et $k$ la constante de raideur du ressort.
                    \1 \underline{Réaction d'un support}
                    \2 Composante normale $\overrightarrow{R_N}$ : perpendiculaire au support.
                    \2 Composante tangentielle $\overrightarrow{R_T} = \overrightarrow{f}$, frottements solides : s'oppose au mouvement.\\Ne dépend que de la masse, $\|\overrightarrow{R_T}\| = \mu_d \|\overrightarrow{R_N}\|$. ($\mu_d$ coefficient de frottements dynamique).\\Quand le système est immobile : $\|\overrightarrow{R_T}\| < \mu_s \|\overrightarrow{R_N}\|$ ($\mu_s$ coef. frottements statique). En général, $\mu_s > \mu_d$.
                    \1 \underline{Forces pressantes}, la somme de des forces pressantes donne la poussée d'Archimède :
                    \\$\overrightarrow{\Pi} = - \rho_{fluide}\ V_{fluide \ d\acute eplac\acute e}\ \vec{g}$
                \end{outline}


        \section{Énergie cinétique et potentielle}\label{sec:energie-cinetique-et-potentielle}

            \twoCol {
                \underline{Énergie cinétique (point matériel)} : $\overrightarrow{E_c} = \dfrac{1}{2} m v^2$\\
                Pour un solide : \[\overrightarrow{E_c} = \sum \dfrac{1}{2} m_i v_i^2\]
            }{
                \underline{Énergie potentielle de position} : \[ E_{pp}(M) = E_{pp}(0) + mgz \]
                Bien définir l'origine des $E_{pp}$\\
                Pour un solide, on prend l'altitude du barycentre.
            }
            \underline{Énergie potentielle élastique} : \[\displaystyle E_{pe}(M) = \dfrac{1}{2} k (x - l_0)^2\]
            Avec $E_{pe} = 0$ pour $x = l_0$.


        \section{Équilibre d'un point matériel}\label{sec:equilibre-d'un-point-materiel}

            \textbf{Seconde loi de Newton : } $\sum \overrightarrow{F_{ext}} = \dfrac{\textrm{d}\vec{p}}{\textrm{d}t} = m \vec{a}$\\
            L'équilibre d'un point s'exprime aussi en fonction de l'énergie potentielle.
            Soit $x_{eq}$ sa position d'équilibre :
            \[ \left( \dfrac{\textrm{d}E_p}{\textrm{d}x}\right) _{x=x_{eq}} = 0\]
            L'équilibre est stable si le point tend à revenir vers $x_{eq}$ s'il s'en est écarté (dérivée seconde positive).


        \section{Équilibre d'un solide}\label{sec:equilibre-d-un-solide}

            $\sum \overrightarrow{F_{ext}} = \vec{0}$ ne suffit pas pour un solide (exemple du couple).\\
            $\rightarrow$ Le moment d'une force caractérise sa capacité à faire tourner un solide.
            \[ \overrightarrow{\mathcal{M}_O}(\overrightarrow{F}) = \overrightarrow{OM} \land \overrightarrow{F} \]

            Bras de levier : distance entre la droite d'action de $\overrightarrow{F}$ et le centre $O$.
            \[ d = \|\overrightarrow{OM}\| \cdot \sin(\overrightarrow{OM}, \overrightarrow{F}) \;\; \text{soit} \;\; \|\overrightarrow{\mathcal{M}_O}(\overrightarrow{F})\| = d\ \|\overrightarrow{F}\| \]

            Dans le plan, le moment est un scalaire, on prend alors le produit scalaire avec un vecteur directeur de l'axe de rotation (le signe du moment détermine alors le sens de rotation).\\

            \textbf{Différents cas possibles :}
            \begin{itemize}
                \item Couple de forces si $\sum_i \overrightarrow{F_i} = \vec 0$ et $\exists\ O\ /\ \sum_i \overrightarrow{M_0}(\overrightarrow{F_i}) \neq \vec 0$ (rotation).\\Le moment d’un couple ne dépend pas du point par rapport auquel il est calculé ($\exists \Rightarrow \forall$).
                \item Solide à l'équilibre si $\sum_i \overrightarrow{F_i} = \vec 0$ et $\exists\ O /\ \sum_i \overrightarrow{M_0}(\overrightarrow{F_i}) = \vec 0$.
                \item Force unique appliquée en $O$ si $\sum_i \overrightarrow{F_i} \neq \vec 0$ et $\sum_i \overrightarrow{M_0}(\overrightarrow{F_i}) = \vec 0$.
                \item Si à la fois la résultante des moments et des forces est non nulle, le solide est à la fois en rotation et en translation.
            \end{itemize}


    \chapter{Cinématique}\label{ch:cinematique}


        \section{Comment repérer la position d’un point ?}\label{sec:comment-reperer-la-position-dun-point-?}

            \twoCol{
                \textbf{Trajectoire du point :}\\
                Rectiligne : ligne droite\\
                Circulaire : arc de cercle\\
                Curviligne : courbe quelconque
            }{
                \textbf{Équation de trajectoire :}\\
                Équation de la forme $y = f(x)$ ou $r = g(\theta)$\\
                Le temps n'intervient plus.
            }


        \section{Vitesse et accélération d’un point matériel}\label{sec:vitesse-et-acceleration-dun-point-materiel}

            \underline{Mouvement uniforme :} $\|\vec{v}\| = const$\\

            \twoCol{
                \textbf{Cilindrique :}\\
                $\vec{OM} = r \vec{e_r} + z \vec{e_z}$, mais $\vec{e_r}$ varie aussi en fonction du temps :\\
                $\dfrac{\re{d}\vec{e_r}}{\re{d}t} = \dot{\theta} \vec{e_\theta}$ \quad et \quad $\dfrac{\re{d}\vec{e_\theta}}{\re{d}t} = - \dot{\theta} \vec{e_r}$\\
                $\vec v = \dot{r} \vec{e_r} + r \dot{\theta} \vec{e_\theta} + \dot{z} \vec{e_z}$
            }{
                \textbf{Base de Frénet :}\\
                $\vec{v} = \|\vec{v}\| \vec{u_T}$\\
                $\vec{a} = \dfrac{d\|\vec{v}\|}{d\text{t}} \vec{u_T} + \dfrac{\|\vec{v}\|^2}{R_C} \vec{u_N}$\\
                Si mouvement uniforme : $\vec{a} = \dfrac{\|\vec{v}\|^2}{R_C} \vec{u_N}$
            }


        \section{Étude de quelques mouvements simples d’un point}\label{sec:etude-de-quelques-mouvements-simples-dun-point}

            \twoCol {
                \textbf{Mouvement rectiligne uniforme :}\\
                $\vec{a} = \vec{0}$\\
                $v(t) = v_0$\\
                $x(t) = v_o t + x_0$
            }{
                Rectiligne accéléré si $\vec v$ et $\vec a$ sont dans le même sens, décéléré sinon.\\
                Rectiligne uniformément accéléré si $\vec a = k \vec{e_x}$ ($k$ constante).
            }
            \\
            \textbf{Mouvement circulaire :}\\
            Vitesse angulaire : $\omega = \dfrac{\re d \theta}{\re d t} = \dot \theta$


        \section{Étude de quelques mouvements simples d’un solide}\label{sec:etude-de-quelques-mouvements-simples-dun-solide}

            \textbf{Translation :} tous les points ont la même vitesse et le même vecteur accélération.\\

            \textbf{Rotation autour d'un point fixe :} tous les points ont la même vitesse angulaire (et le même vecteur accélération angulaire).


        \section{Changement de référentiels}\label{sec:changement-de-referentiels}

            Soit deux référentiels $R$ et $R'$, de centres $O$ et $O'$, on a : $\vec{OM} = \vec{OO'} + \vec{O'M}$

            On a alors une position absolue ($\vec{OM}$) et relative ($\vec{O'M}$).
            De même pour la vitesse.\\

            Si on dérive par rapport au référentiel $R$, il faudra dériver les vecteurs unitaires du référentiel $R'$ et vice versa.\\

            Soit $\vec{OM} = \vec{OO'} + x' \vec{e_{x'}} + y' \vec{e_{y'}} + z' \vec{e_{z'}}$\\
            Alors $\vec{v_{R}} = \vec{v_{R'}} + \vec{v_{R'/R}}(M) = \left( \dfrac{\re d \vec{OO'}}{\re d t} \right)_R + \dfrac{\re d x'}{\re d t} \vec{e_{x'}} + \dfrac{\re d y'}{\re d t} \vec{e_{y'}} + \dfrac{\re d z'}{\re d t} \vec{e_{z'}} + x' \left( \dfrac{\re d \vec{e_{x'}}}{\re d t} \right)_R + y' \left( \dfrac{\re d \vec{e_{y'}}}{\re d t} \right)_R + z' \left( \dfrac{\re d \vec{e_{z'}}}{\re d t} \right)_R$\\

            Dérivée de $\vec{OO'}$ = translation.
            Dérivées des vecteurs unitaires = rotation : peut être retiré en cas de translation.\\
            En translation, l'accélération vaut :\\

            $\vec{a_R} = \vec{a_{R'}} + \left( \dfrac{\re{d}^2 \vec{OO'}}{\re d t^2} \right)_R$


    \chapter{Introduction à la dynamique}\label{ch:introduction-a-la-dynamique}

        À Chaque fois :
        \begin{itemize}
            \item Grandeur d'inertie : capacité à résister à la modification du mouvement.
            \item Grandeur cinématique : décrit le mouvement.
        \end{itemize}


        \section{Les grandeurs cinétiques du point}\label{sec:les-grandeurs-cinetiques-du-point}

            \twoCol {

                \subsection{Énergie cinétique}\label{subsec:l-energie-cinetique}

                $\varepsilon_c = \dfrac{1}{2} m v^2$\\
                ($m$ grandeur d'inertie et $v$ grandeur cinématique).

                \[\textbf{PFD : } \dfrac{\re d \varepsilon_c}{\re d t} = \summ P_{ext}\]
            }{

                \subsection{Quantité de mouvement}\label{subsec:la-quantite-de-mouvement}

                $\vec p = m \vec v$\\
                ($m$ grandeur d'inertie et $v$ grandeur cinématique).\\
                \vspace{-14pt}
                \begin{align*}
                    \textbf{PFD :   } \dfrac{\re d \vec p}{\re d t} &= \summ \vec{F_{ext}}\\
                    m \vec a &= \sum \vec{F_{ext}}
                \end{align*}
            }

            \subsection{Moment cinétique}\label{subsec:le-moment-cinetique}
                \threeCol[15][45] {
                \vspace{-18pt}
                \begin{flalign*}
                    \vec \sigma &= \vec{OM} \wedge \vec p&\\
                    &= \vec{OM} \wedge m \vec v&\\
                    &= r \vec{e_r} \wedge m r \dot\theta \vec{e_\theta}&\\
                    &= m r^2 \dot\theta \vec{e_z}&\\
                    &= J \vec \Omega&
                \end{flalign*}
            }{
                \begin{itemize}
                    \item Moment d'inertie : $J = m r^2$\\(grandeur d'inertie)
                    \item Vecteur rotation instantané : $\vec \Omega = \dot\theta \vec{e_z}$\\(grandeur cinématique)
                \end{itemize}
            }{
                $\textbf{PFD : } \dfrac{\re d \vec \sigma}{\re d t} = \summ \vec{\mathcal{M}_{ext}}$
            }


        \section{Les grandeurs cinétiques d'un solide}\label{sec:les-grandeurs-cinetiques-d'un-solide}

            \subsection{Énergie cinétique d'un solide}\label{subsec:l'energie-cinetique-d'un-solide}

                Somme des énergies cinétiques des points du solide : $\varepsilon_c = \summ \varepsilon_{c,i}$\\

                \twoCol[45] {
                \underline{En translation :}\\
                Tous les points ont la même vitesse :\\ $\varepsilon_c = \dfrac{1}{2} m v^2$
            }{
                \underline{En rotation :}\\
                $v = r \dot\theta$ donc $\varepsilon_c = \dfrac{1}{2} \dot\theta^2 \summ{m_i r^2_i} = \dfrac{1}{2} \dot\theta^2 J_\Delta$\\

                Moment d'inertie par rapport à l'axe $\Delta$ : $J_\Delta = \intt \( r^2 \re d m \)$
            }

                \twoCol[45] {
                \subsection{Quantité de mouvement d’un solide}\label{subsec:la-quantite-de-mouvement-dun-solide}
                (Solide en translation)\\
                $\vec p = \summ{m_i \vec v_i} = M \vec v$
            }{
                \subsection{Moment cinétique d'un solide}\label{subsec:le-moment-cinetique-dun-solide}
                Solide en rotation autour d'un axe fixe $\Delta$ à vitesse $\vec \Omega = \dot\theta \vec{u_z}$\\
                $\vec \sigma = \vec \Omega \summ{J_i} = J \vec \Omega$
            }


        \section{Le principe d'inertie}\label{sec:le-principe-d'inertie}

            Un système est incapable de changer son état par lui même.\\
            Dans un référentiel Galiléen, tout système isolé ou pseudo-isolé est immobile ou à vitesse uniforme.\\

            On dit que les grandeurs cinétiques d'un système isolé ou pseudo-isolé sont conservatives :\\
            \[\dfrac{\re d \varepsilon_c}{\re d t} = 0 \quad\quad \dfrac{\re d \vec p}{\re d t} = \vec 0 \quad\quad \dfrac{\re d \vec \sigma}{\re d t} = \vec 0\]\\


    \chapter{Dynamique : Lois de Newton}\label{ch:dynamique-lois-de-newton}


        \section{Dynamique du point}\label{sec:dynamique-du-point}

            \subsection{Les trois lois du mouvement}\label{subsec:les-trois-lois-du-mouvement}

                \begin{itemize}
                    \item \underline{\textbf{1ère loi de Newton : principe d’inertie}}\\
                    Il existe des référentiels privilégiés dits galiléens dans lesquels le mouvement d'un point isolé ou pseudo-isolé est rectiligne uniforme.\\
                    Ainsi, si $R_g$ est galiléen, alors tout référentiel en translation rectiligne uniforme par rapport à $R_g$ est aussi galiléen.\\

                    \item \underline{\textbf{2ème loi de Newton : principe fondamental de la dynamique (PFD)}}\\
                    Les forces sont à l’origine du mouvement : $\summ \vec F_\re{ext} = \dfrac{\re \vec d p}{\re d t} = m \vec a$ (dans un ref. galiléen).\\

                    \item \underline{\textbf{3ème loi de Newton : principe des actions réciproques}}\\
                    Une force unique existant seule n'existe pas.\\
                    Si deux points sont en interaction, on a deux forces : $\vec{f_\re{1/2}} = - \vec{f_\re{2/1}}$\\
                    D'où la résultante des forces intérieures à un système est nulle.
                \end{itemize}

            \subsection{Théorème du moment cinétique}\label{subsec:theoreme-du-moment-cinetique}

                TMC pour un point matériel en rotation autour d'un axe fixe $\Delta$ :
                \[\sum \mathcal{M}_{\vec F} (\Delta) = J_\Delta \ddot \theta = J_\Delta \dot \omega \]
                ($J_\Delta \ddot \theta$ est la dérivée du moment cinétique projeté sur $\Delta$)\\


        \section{Dynamique des mouvements simples du solide}\label{sec:dynamique-des-mouvements-simples-du-solide}

            \subsection{Introduction}\label{subsec:introduction}

                \twoCol {
                \subsubsection{Solide en translation}\label{subsubsec:solide-en-translation}
                Tous les vecteurs interne restent équipollents\\
                (tous les point on la même vitesse et acceleration)
            }{
                \subsubsection{Rotation autour d'un axe fixe}\label{subsubsec:rotation-autour-d'un-axe-fixe}
                Tous les points ont la même vitesse angulaire :\\
                Vitesse angulaire $\omega = \dot \theta$\\
                Vecteur rotation instantanée $\vec \Omega = \omega\ \vec{u_z}$\\
                Le vecteur vitesse peut être retrouvé par $\vec v = \vec \Omega \wedge \vec{OM}$
            }
                \vspace{12pt}\\
                \twoCol {
                \subsection{Solide en translation}\label{subsec:solide-en-translation}

                La quantité de mouvement d’un solide est égale à la quantité de mouvement de son barycentre, affecté de la masse totale du système : $\vec{P} = M\ \vec{v_G}$\\
                \textbf{PFD :} $\summ \vec{F_{ext}} = M\ \vec{a_G} = \dfrac{\re d \vec{P}}{\re d t}$\\
            }{
                \subsection{Solide en rotation autour d'un axe fixe}\label{subsec:solide-en-rotation-autour-d'un-axe-fixe}

                \textbf{TMC :} $\summ \Gamma_{\vec{F}_\re{ext}} (\mathrm{Oz}) = J_\re{Oz}\ \ddot \theta$\\

                Avec $\Gamma_{\vec{F}} (\re{Oz}) = rF$ (si force selon $\vec {e_\theta}$) \vspace{4pt}\\
                Et $J_\re{Oz} = m r^2$ (si masse uniforme)\\
            }


    \chapter{Dynamique : Théorèmes énergétiques}\label{ch:dynamique-theoremes-energetiques}


        \section{Travail d’une force}\label{sec:travail-dune-force}

            Le travail mécanique représente la quantité d’énergie échangée entre le système et le milieu extérieur et pouvant être transformée d’une forme en une autre.\\

            Pour une force constante, $W_{A\scriptveryshortarrow B}(\vec F) = \vec F \cdot \vec{AB}$\\

            On note aussi, $\delta W = \vec F \cdot \vec l$, ce qui donne pour une force quelconque :\\
            \[ W_{A\scriptveryshortarrow B}(\vec F) = \int_A^B \vec F \cdot \vec{\re dl} \]

            On définis de même la puissance d'une force :\\
            \[ P(\vec F) = \dfrac{\delta W(\vec F)}{\re d t} = \vec F \cdot \dfrac{\vec{\re d l}}{\re dt} = \vec F \cdot \vec v \]


        \section{Théorème de l'énergie cinétique (TEC)}\label{sec:theoreme-de-l'energie-cinetique-(tec)}

            Dans un référentiel galiléen, la variation de l'énergie cinétique d'un point matériel est égale à la somme des travaux de toutes les forces :

            \[\Delta E_C = E_C(B) - E_C(A) = \sum_i W_{A \scriptveryshortarrow B}(\vec{F_i}) = W_{A \scriptveryshortarrow B}(\vec{F_{ext}})\]


        \section{Énergie potentielle}\label{sec:energie-potentielle}


            \begin{itemize}
                \item \underline{Force conservative :} son travail dépend uniquement des positions initiales et finales.\\
                On dit que la force dérive d'une énergie potentielle : $\re d E_P = \vec{\operatorname{grad}}\ E_P \cdot \vec{d l} = -\re d W$ donc $\vec F = - \vec{\operatorname{grad}}\ E_P$\\
                \item \underline{Force non conservative :} son travail dépend du chemin suivi, elles n'ont pas d'énergie potentielle associée.
            \end{itemize}


        \section{Énergie mécanique}\label{sec:energie-mecanique}

            Définition de l'énergie mécanique :
            \[E_M = E_C + \sum E_P\]

            Dans un référentiel galiléen, la variation d’énergie mécanique d’un point matériel est égale à la somme des travaux des forces non-conservatives :

            \[\Delta E_M = E_M(B) - E_M(A) = \sum_i W_{A \scriptveryshortarrow B}(\vec{F_{i,\textbf{nc}}}) = W_{A \scriptveryshortarrow B}(\vec{F_{ext,\textbf{nc}}})\]

\end{document}
