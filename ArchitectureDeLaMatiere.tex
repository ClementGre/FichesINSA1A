\documentclass[13pt, twoside, a4paper, french, tikz]{report}

\usepackage{ficheslib}
\usepackage{booktabs}
\newcommand*{\getSubject}{Architecture de la matière}

\begin{document}
\title{\getSubject}
\author{Clément GRENNERAT}
\date{Septembre 2022}


\chapter{Le modèle quantique}\label{ch:le-modele-quantique}
  
  
  \section{Les interactions radiation électromagnétique - matière}\label{sec:les-interactions-radiation-electromagnetique---matiere}
    
    Radiation électromagnétique : forme de déplacement d'énergie dans l'espace.\\
    
    \textbf{Aspect ondulatoire :} onde caractérisé par : sa vitesse $c$, sa fréquence $\nu$, sa longueur d'onde $\lambda$, son amplitude\ldots\\
    
    \resizebox{\textwidth}{!}{
      \includestandalone[mode=buildnew,page=1, scale]{out/electromagnetic_spectrum}
    }
    
    \textbf{Aspect corpusculaire :} photon associé à une radiation de longueur d'onde $\nu$, tel que $E = h . \nu$
    
    \subsection{Mise en oeuvre de la spectroscopie}\label{subsec:mise-en-oeuvre-de-la-spectroscopie}
      
      \twoCol{
      \textbf{Pour l'émission}, on a dans l'ordre : source à analyser, fente, achromat (lentille convergente), prisme, achromat, écran.
      On peut avoir : un spectre continu (ex. lumière visible), ou un spectre de raies (ex. vapeur métallique excitée).\\
      
      (Avec $n' \le n$, car descend du niveau $n$ à $n'$), Formule de Ritz-Balmer ou de Rydberg :
      \[\dfrac{1}{\lambda} = R_{x} Z^2 \left(\dfrac{1}{n'^2} - \dfrac{1}{n^2}\right)\]
      
      \makebox[3cm][l]{$n' = 1$ : Lyman} \;\; $n' = 2$ : Balmer\\
      \makebox[3cm][l]{$n' = 3$ : Paschen} \;\; $n' = 4$ : Brackett
      
    }{
      \textbf{Pour l'absorption}, on place une source de lumière blanche, puis on place la cuve de la substance absorbante avant le prisme. On a alors un spectre discontinu.\\
      
      (Part toujours de l'état fondamental, monte jusqu'au niveau $n$)
      \[\dfrac{1}{\lambda} = R_{x} Z^2 \left(1 - \dfrac{1}{n^2}\right)\]
      
    }
      
      \vspace*{10pt}
      Pour les hydrogénoïdes, on utilise $Z$ (numéro athomique) et $R_x \approx R_H$.\\
      
      Raie de tête : du niveau $n'+1 \rightarrow n'$ \;\;\; Raie limite : du niveau $\infty\rightarrow n'$
  
  
  \section{Interprétation théorique : le modèle de Bohr pour l'atome d'hydrogène}\label{sec:interpretation-theorique--le-modele-de-bohr-pour-latome-dhydrogene}
    
    \textbf{Trois hypothèses :}
    \begin{itemize}
      \item L'électron gravite autour du noyau sur une orbite circulaire de rayon $r$, à vitesse $v$.
      \item Le rayon de Bohr $r$ est quantifié (seuls certaines valeurs possibles).
      \item Si $r$ est constant, l'élection ne rayonne pas d'énergie (état stationnaire).
    \end{itemize}
    \textbf{Limites :}
    \begin{itemize}
      \item Raie = ensemble de sous raies très fines.
      \item Spectre d'émission modifié dans un champ magnétique ou électrique intense.
      \item Ne fonctionne pas pour les atomes polyélectroniques.
    \end{itemize}
  
  
  \section{Présentation des résultats : diagramme d'énergie de Grotrian}\label{sec:presentation-des-resultats--diagramme-denergie-de-grotrian}
    
    L'énergie d'un niveau s'exprime à partir de celle du niveau 1 : $E_n = -\dfrac{E_1}{n^2}$ (pour les Hydrogénoïdes).\\
    L'énergie d'ionisation correspond à la transition $1 \rightarrow \infty$.\\
    On a donc $E_{incident} = E_{ionisation} + E_c$


\chapter{Le modèle ondulatoire de l'atome}\label{ch:le-modele-ondulatoire-de-l'atome}
  
  
  \section{Édification de la mécanique quantique}\label{sec:edification-de-la-mecanique-quantique}
    
    Modèles de l'atome non viables $\rightarrow$ mécanique quantique, propre aux atomes et molécules (tout système influençé par un quantum $h.\nu$).
  
  
  \section{Particule ou onde ? Relation de Bröglie}\label{sec:particule-ou-onde-?-relation-de-broglie}
    
    Dualité onde-corpuscule (onde, caractérisé par $\nu$, $\lambda$, $c$ ; corpuscule, caractérisé par $E = h . \nu = \dfrac{h.c}{\lambda}$).\\
    Relation de Bröglie : $\lambda = \dfrac{h}{m . v} = \dfrac{h}{p}$\\
    Pour les photons (masse nulle), on utilise ka quantité de mouvement $p$.
  
  
  \section{De la notion d'orbite à la notion d'orbitale}\label{sec:de-la-notion-dorbite-a-la-notion-dorbitale}
    
    Principe d'indétermination d'Heisenberg : dans la théorie quantique, la position et la vitesse ne sont pas déterminés, on parle donc de densité électronique (probabilité de présence).
    La densité électronique définie une orbitale.\\
    
    L'équation de Schrödinger lie l'énergie d'un électron et sa probabilité de présence en différents points, avec la fonction d'onde $\Psi(x, y, z)$.
    \begin{itemize}
      \item La probabilité de présence ne peut prendre qu'une seule valeur par point de l'espace.
      \item La probabilité de présence ne peut pas présenter de discontinuité.
      \item En explorant tout l'espace, la probabilité de présence doit être égale à 1 (évènement certain).
    \end{itemize}
    
    \vspace{7pt}
    \textbf{Nombres quantiques :}
    \vspace{7pt}
    
    \begin{tabular}{|c|c|c|c|}
      \hline
      \textbf{Symbole} & \textbf{Nom}             & \textbf{Représentation}          & \textbf{Valeurs possibles}         \\\hline
      $n$              & principal                & Taille de l'orbitale             & $n \ge 1$                          \\\hline
      $l$              & secondaire (ou azimutal) & Type d'orbitale                  & $0 \le l \le n-1$                  \\\hline
      $m_l$ (ou $m$)   & magnétique ou orbital    & Orientation de l'orbitale        & $-l \le m_l \le l$                 \\\hline
      $s$ (ou $m_s$)   & spin                     & Rotation de l'$e^-$ sur lui même & $+\sfrac{1}{2}$ ou $-\sfrac{1}{2}$ \\\hline
    \end{tabular}\label{tab:table}
    
    \vspace{7pt}
    \textbf{Différentes orbitales :}
    \vspace{7pt}
    
    \begin{tabular}{|c|c|c|l|}
      \hline
      \textbf{$l$} & \textbf{Nom} & \textbf{Origine} & \textbf{Forme}                \\\hline
      0            & $s$          & sharp            & Sphérique                     \\\hline
      1            & $p$          & principal        & Deux ellipsoïdes, ou haltères \\\hline
      2            & $d$          & diffuse          &                               \\\hline
      3            & $f$          & fundamental      &                               \\\hline
    \end{tabular}


\chapter{Atomes polyélectroniques,\\configuration électronique}\label{ch:atomes-polyelectroniques---configuration-electronique}
  
  
  \section{Approximation monoélectronique. Charge nucléaire effective}\label{sec:approximation-monoelectronique.-charge-nucleaire-effective}
    
    Pour les atomes à plusieurs électrons, les spectres d'absorption et d'émission sont nettement plus complexes, et l'équation de Schrödinger est trop complexe pour être résolue.\\
    
    On fait donc une approximation monoélectronique en remplaçant l'ensemble {noyau + autres électrons} par un noyau fictif de \textbf{charge nucléaire effective} $Z^*$.\\
    Les autres électrons exercent sur un électron particulier, un \textbf{effet d'écran} représenté par la constante d'écran $\sigma$ : $Z^* = Z - \sigma$.
  
  
  \section{Orbitales atomiques}\label{sec:orbitales-atomiques}
    
    Pour les atomes polyélectroniques, les énergies dépendent de $n$, de $l$ et de $Z^*$.
  
  
  \section{Organisation du nuage électronique}\label{sec:organisation-du-nuage-electronique}
    
    \twoCol[50] {
      
      Couche électronique (ou période) :\\
      
      \begin{tabular}{lllllllL}
        \toprule
        \textbf{n}       & \textbf{1} & \textbf{2} & \textbf{3} & \textbf{4} & \textbf{5} & \textbf{6} & \textbf{7} \\
        \midrule
        Symbole          & K          & L          & M          & N          & O          & P          & Q          \\
        \midrule
        Valeurs de $l$   & 1          & 2          & 3          & 4          & 4          & 3          & 2          \\
        \midrule
        Valeurs de $m_l$ & 1          & 4          & 9          & 14         & 14         & 9          & 4          \\
        \midrule
        Électrons        & 2          & 8          & 8          & 18         & 18         & 32         & 32         \\
        \bottomrule
      \end{tabular}
      \vspace{7pt}\\
      Dans une période $n$, il y a $n^2$ valeurs de $m_l$ possibles, soit $2n^2$ électrons, mais l'ordre de remplissage modifie le nombre d'électrons dans chaque période\\
    }{
      Sous-couches :\\
      
      \begin{tabular}{lllllll}
        \toprule
        \textbf{l}       & \textbf{0} & \textbf{1} & \textbf{2} & \textbf{3} & \textbf{4} & \textbf{5} \\
        \midrule
        Symbole          & $s$        & $p$        & $d$        & $f$        & $g$        & $h$        \\
        \midrule
        Valeurs de $m_l$ & 1          & 3          & 5          & 7          &            &            \\
        \midrule
        Valeurs de $m_s$ & 2          & 6          & 10         & 14         &            &            \\
        \bottomrule
      \end{tabular}
      \vspace{7pt}\\
      Dans une sous-couche $l$, il y a $2l + 1$ valeurs de $m_l$ possibles, soit $2(2l+1)$ électrons.
    }
    
    Principe de Pauli : un atome ne peut pas avoir deux électrons avec les quatre mêmes nombres quantiques (trivial).\\
    
    \twoCol[55]{
      
      \subsection{Règle de Klechkowski}\label{subsec:regle-de-klechkowski}
      
      Le remplissage se fait à $(n+l)$ croissant, et à $n$ croissant en cas d'égalité :\\
      
      Ordre de remplissage des couches :\\
      
      \begin{tabular}{cc|llll|llll}
        \toprule
        \multirow{2}{*}{Couche} & \multirow{2}{*}{n} & \multicolumn{4}{c|}{Sous-couche} & \multicolumn{4}{c}{\multirow{2}{*}{Remplissage}} \\
        &   & l=0  & 1    & 2    & 3    &      &      &      &      \\
        \midrule
        K & 1 & $1s$ &      &      &      & $1s$ &      &      &      \\
        L & 2 & $2s$ & $2p$ &      &      & $2s$ &      &      & $2p$ \\
        M & 3 & $3s$ & $3p$ & $3d$ &      & $3s$ &      &      & $3p$ \\
        N & 4 & $4s$ & $4p$ & $4d$ & $4f$ & $4s$ &      & $3d$ & $4p$ \\
        O & 5 & $5s$ & $5p$ & $5d$ & $5f$ & $5s$ &      & $4d$ & $5p$ \\
        P & 6 & $6s$ & $6p$ & $6d$ &      & $6s$ & $4f$ & $5d$ & $6p$ \\
        Q & 7 & $7s$ & $7p$ &      &      & $7s$ & $5f$ & $6d$ & $7p$ \\
        \bottomrule
      \end{tabular}
    }{
      \subsection{Règle de Hund}\label{subsec:regle-de-hund}
      
      Toutes les orbitales d'une sous-couche doivent être occupées chacune par un électron célibataire avant que l'une d'elles puisse être occupé par deux électrons appariés (on remplis d'abord toutes les cases quantiques de la sous-couche avant d'apparier les électrons de spin opposé).\\
      
      \subsection{Exceptions aux règles de remplissage}\label{subsec:exceptions-aux-regles-de-remplissage}
      
      La sous-couche $d$ est particulièrement stable lorsqu'elle est pleine ou remplie à moitié (5 ou 10 électrons)
      \begin{itemize}
        \item Le chrome (Cr) a 5 électrons $3d$ et seulement unn électron $4s$
        \item Le cuivre (Cu) a dix électrons $3d$ et seulement un électron $4s$
      \end{itemize}
    }


\chapter{La Classification Périodique des Éléments}\label{ch:la-classification-periodique-des-elements}
  
  
  \section{Présentation}\label{sec:presentation}
    
    Classement des éléments par numéro atomique $Z$ croissant.\\
    Un peu plus de 100 éléments dont 90 existent naturellement.
  
  
  \section{Construction et description}\label{sec:construction-et-description}
    
    \resizebox{\textwidth}{!}{
      \includestandalone[mode=buildnew,page=1, scale]{out/periodic_table}
    }
    
    \begin{itemize}
      \item Période ou couche : ligne horizontale, correspond au nombre quantique $n$
      \item Groupe ou famille : colonne verticale (propriétés chimiques semblables).
      \item Bloc : correspond au nombre quantique $l$ (bloc $s$, $p$, $d$, et $f$)
    \end{itemize}
    
    \vspace{7pt}
    
    Une couche est pseudo-saturée si la couche est bel et bien pleine, mais qu'il reste la sous-couche $d$ ou $f$ à remplir.\\
    La couche $n=3$ est pseudo-saturée si $s$ et $p$ sont remplies, mais pas $d$.
  
  
  \section{Familles principales}\label{sec:familles-principales}
    
    \twoCol[70]{
      \vspace{-7pt}\\
      \begin{tabular}{l|lll}
        \toprule
        Famille                             & Position              & Structure    & Couleur      \\
        \midrule
        \textbf{Alcalins}                   & groupe 1              & $ns^1$       & Bleu         \\
        \textbf{Alcalino-terreux}           & groupe 2              & $ns^2$       & Violet foncé \\
        \textbf{Halogènes}                  & groupe 17             & $ns^2\ np^5$ & Jaune        \\
        \textbf{Gaz rares ou nobles}        & groupe 18             & $ns^2\ np^6$ & Vert         \\
        \textbf{Lanthanides (terres rares)} & bloc $f$, période $6$ & $6s^2\ 4f^x$ & Rouge        \\
        \textbf{Actinides (terres rares)}   & bloc $f$, période $7$ & $7s^2\ 5f^x$ & Rouge        \\
        \bottomrule
      \end{tabular}\\
    }{
      \subsection{Métaux de transition}\label{subsec:metaux-de-transition}
      
      Ce sont les éléments du groupe 3 à 11 (bloc $d$, dernière colonne exclue).\\
      Les métaux de transition interne sont les lanthanides et les actinides (terres rares, bloc $f$, deux dernières colonnes exclues).
    }
    
    \vspace{4pt}\\
    
    Les éléments sont métalliques jusqu'à la couche 12.
    Le bloc $d$ se divise ensuite en diagonale selon la \textbf{règle de Sanderson} :\\
    L'élément sera métallique si le nombre d'électrons $N_e$ sur sa couche de $n$ le plus élevé ($ns^a\ np^b$ avec $N_e = a + b$) est inférieur ou égal au numéro de la période.\\
    Les éléments métalliques ont tendance à former des ions positifs (pour avoir la configuration électronique du gaz inerte qui les précède).


\chapter{Propriétés physiques des éléments}\label{ch:propriete-physique-des-elements}
  
  
  \section{Charge nucléaire effective}\label{sec:charge-nucleaire-effective}
    
    \textbf{Règles de slater} (empiriques) permettent de calculer la constante d'écran $\sigma_i$ et la charge nucléaire effective $Z^* = Z - \sigma_i$ agissant sur le $i^{\text{ème}}$ électron.\\
    
    \begin{itemize}
      \item On écrit la configuration électronique de l'élément sous forme de groupe : \[(1s)(2s, 2p)(3s, 3p)(3d)(4s, 4p)(4d)(4f)(5s, 5p)(5d)\ \ldot\]
      \item Les électrons du même groupe apportent une contribution de $0,35$ et $0,30$ pour la couche $1s$.
      \item Les électrons des groupes de gauche apportent chacun une contribution dépendant des groupes et périodes relatives.
    \end{itemize}
  
  
  \section{Rayon atomique}\label{sec:rayon-atomique}
    
    C'est la moitié de la distance séparant deux atomes engagés dans une liaison simple.
    \begin{itemize}
      \item Dans une période donnée (ligne, $n$ fixé), le rayon atomique est \textbf{décroissant} (Diminue quand $Z$ augmente).
      \item Dans un groupe donné (colonne), le rayon atomique est \textbf{croissant} (Augmente avec $Z$).
    \end{itemize}
    \vspace{7pt}
    En effet, plus Z est grand, plus il y a de protons qui attirent les électrons.\\
    Mais quand $n$ augmente, une nouvelle orbite plus large est ajoutée : le rayon atomique augmente d'un coup.
  
  
  \section{Énergie d'ionisation}\label{sec:energie-d'ionisation}
    
    Énergie correspondant à la réaction $\ce{A_{(gaz)}} \longrightarrow \ce{A^+_{(gaz)}} + \ce{e^-}$
    
    \begin{itemize}
      \item Dans une période donnée (ligne, $n$ fixé), l'énergie d'ionisation est \textbf{croissante} (Augmente avec $Z$).
      \item Dans un groupe donné (colonne), l'énergie d'ionisation est \textbf{décroissante} (Diminue quand $Z$ augmente).
      \item Pour les élément de transition, l'énergie d'ionisation augmente moins vite.
    \end{itemize}
    \vspace{7pt}
    
    Évolue dans le sens inverse du rayon atomique : plus Z est grand, plus l'atome se rétrécit, donc il est plus difficile de lui retirer un électron.\\
    Mais quand $n$ augmente, une nouvelle orbite plus large est ajoutée, avec des électrons plus facile à retirer.
  
  
  \section{Électronégativité}\label{sec:electronegativite}
    
    On utilise l'échelle de Pauling : l'électronégativité s'exprime en fonction des propriétés des molécules diatomiques.
    
    \begin{itemize}
      \item Pour les groupes 1, 2, 13 à 18 (bloc $s$ et $p$), l'électronégativité augmente d'un groupe à l'autre (de gauche à droite), et décroit d'une période à l'autre (de haut en bas).
      \item L'électronégativité varie peu dans les éléments de transition (groupe 3 à 12, bloc $d$), mais sera maximale en bas à droite, et minimale en bas à gauche.
    \end{itemize}
    \vspace{7pt}
    
    Le sens de variation est le même que pour l'énergie d'ionisation, excepté pour le bloc $d$.


\chapter{Spectroscopie des rayons X}\label{ch:spectroscopie-des-rayons-x}
  
  
  \section{Les rayons X : nature et production}\label{sec:les-rayons-x-:-nature-et-production}
    
    Radiations inférieures à $10 nm$. Rayons X de spectroscopie/radiocristallographie : $10 nm$ (quelques $keV$).
    
    On produit les rayons X par accélération d'électrons : ils partent de la cathode (filament chauffé), et arrivent sur l'anticathode ou anode.
  
  
  \section{Spectre d'émission des rayons X}\label{sec:spectre-d'emission-des-rayons-x}
    
    \subsection{Fond continu}\label{subsec:fond-continu}
      
      $\Rightarrow$ Émis par l'électron ralentis dans la cible.
      L'énergie des rayons X produits ne peut pas dépasser l'énergie de l'électron incident, on a donc $\dfrac{hc}{\lambda e} \le U$ (avec U la tension d'accélération).\\
      Il vient $\lambda \ge \dfrac{hc}{eU}$.
      Le fond continu n'est donc présent qu'après une certaine longueur d'onde $\lambda_{0}$ (seuil).
    
    \subsection{Le spectre de raies caractéristiques}\label{subsec:le-spectre-de-raies-caracteristiques}
      
      Lors de l'expérience de spectroscopie, un électron de couche interne est arraché et une réorganisation électronique emmet des raies caractéristiques de l'élément constituant l'anticathode.
      Séries de raies : K, L, M, N\ldots\\
      
      \textbf{Couplage spin-orbite} : cinquième nombre quantique $j = \bigg | l \pm \dfrac{1}{2} \bigg |$.
      Les niveaux $p$, $d$, et $f$ sont donc dédoublés.\\
      
      \textbf{Règles de selection} : $\Delta l = \pm 1$ et $\Delta j = 0$ ou $\pm 1$.
      De plus, les probabilités de transitions intra couche sont tellement faibles qu'elles ne seront pas considérées.\\
      
      Les transitions sont identifiées par deux lettres et éventuellement un indice, par exemple : $K$-$L_2$ ou $L_2$-$M_1$.
  
  
  \section{La loi de Moseley pour l'émission des rayons X}\label{sec:la-loi-de-moseley-pour-l'emission-des-rayons-x}
    
    La fréquence des raies est liée au numéro atomique par la loi de Moseley : $\sqrt{\nu} = a(Z-b)$, $a$ et $b$ constantes pour une raie donnée (mais garder un domaine de valeurs de $Z$ faible pour limiter les écarts à la linéarité).\\
    Ceci découle de la généralisation de la formule de Ritz-Balmer : $\nu_{n\rightarrow n'} = c R_X (Z - \sigma)^2\left(\dfrac{1}{n'^2} - \dfrac{1}{n^2}\right)$.
  
  
  \section{Atténuation des rayons X}\label{sec:attenuation-des-rayons-x}
    
    On place un filtre devant le faisceau de rayons X : les photons interagissent alors avec les atomes par effet photoélectrique et diffusion.
    \begin{itemize}
      \item \underline{Interaction photoélectrique} : photon absorbé et électron arraché : $E_{\text{inscident}} = E_{\text{électron}} + E_c$.
      Lors de la réorganisation des couches, est émis un rayonnement de fluorescence caractéristique du filtre.
      \item \underline{Diffusion} : photon absorbé, un électron absorbe une partie de l'énergie et est éjecté, et le reste de l'énergie contribue à l'émission d'un photon secondaire.
    \end{itemize}
    \vspace{7pt}
    \twoCol {
      
      \subsection{Loi de Beer Lambert}\label{subsec:loi-de-beer-lambert}
      
      $I = I_0 \exp(-\mu l)$ avec $\mu$ le coefficient d'atténuation linéique.
      Dans le cas d'un faisceau large, il faut ajouter à l'intensité transmise, celle du rayonnement diffusé.
      \vspace{1pt}\\
      Plus $Z$ de l'élément absorbant est faible plus $\mu$ est petit, et plus $\lambda$ de la radiations est faible (soit $E$ grand), plus $\mu$ est petit.
    }{
      
      \subsection{Monochromatisation d'un faisceau de RX}\label{subsec:monochromatisation-d'un-faisceau-de-rx}
      
      Le coefficient d'atténuation linéique décroit brusquement quand $\lambda$ dépasse la longueur d'onde d'un niveau (car l'énergie n'est plus suffisante pour arracher les électrons du niveau). On peut donc utiliser ces discontinuités pour monochromatiser le faisceau.\\
      Par exemple pour absorber la raie $K$-$L$ on doit avoir :
      \vspace{3pt}\\
      $\lambda_{K\text{-}M\text{ (anticathode)}} < \lambda_{K \text{ (absorbant)}} < \lambda_{K\text{-}L\text{ (anticathode)}}$
      \vspace{3pt}\\
      Généralement, le métal de $Z' = Z-1$ peut satisfaire.
    }
    

\end{document}